\begin{definition}[Inverso módulo $m$]
    Sejam $a,a',m\in\ints$. Chamamos $a'$ o inverso (multiplicativo) de $a$ módulo $m$ sse
    $$
        aa'\cong 1\pmod m.
    $$
    \begin{itemize}[--]
        \item O inverso é único;
        \item O $a$ tem inverso módulo $m$ sse $\gcd a b = 1.$
    \end{itemize}
\end{definition}

\begin{exercise}
\topic{congruência}
    Prove a unicidade do inverso módulo $m$. 
\end{exercise}

\begin{exercise}
\topic{congruência!teorema!Chinês do resto}
    Usando o Teorema Chinês do Resto, ache todos os inteiros $x\in\ints$ que satisfazem o sistema de congruências:
    $$
    \left\{
        \begin{aligned}
            x&\cong 2\pmod 9\\
            x&\cong 1\pmod 5\\
            x&\cong 2\pmod 4
        \end{aligned}
    \right.
    $$
\end{exercise}

\begin{exercise}
\topic{congruência!teorema!Chinês do resto}
    Ache todos os inteiros $x\in\ints$ com $\abs x < 64$ que satisfazem o sistema de congruências:
    $$
    \left\{
        \begin{aligned}
            x &\cong 1\pmod 3\\
            3x&\cong 1\pmod 4\\
            4x&\cong 2\pmod 5
        \end{aligned}
    \right.
    $$
\end{exercise}

\begin{homework}
\topic{congruência!teorema!Chinês do resto}
    Use o Teorema Chinês do Resto para resolver os sistemas de congruências:
    $$
    \text{(a)}
    \left\{
            \begin{aligned}
                x&\cong 1\pmod 3\\
                x&\cong 2\pmod 5\\
                x&\cong 3\pmod 7
            \end{aligned}
    \right.
    \qquad
    \text{(b)}
    \left\{
            \begin{aligned} 
                x&\cong 2\pmod {\phantom 04}\\
                x&\cong 3\pmod {\phantom 05}\\
                x&\cong 4\pmod {\phantom 09}\\
                x&\cong 5\pmod {13}
            \end{aligned}
    \right.
    \qquad
    \text{(c)}
    \left\{
            \begin{aligned}
                x&\cong 3\pmod 4\\
               5x&\cong 1\pmod 7\\
                x&\cong 2\pmod 9
            \end{aligned}
    \right.
    $$
\end{homework}

\begin{homework}
    Encontre o menor inteiro positivo que deixa os restos $8$, $7$ e $11$ quando dividido por $7$, $11$ e $15$, respectivamente.
\end{homework}
