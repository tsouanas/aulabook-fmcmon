\begin{exercise}
\topic{teoria de números}
	Se $a < 0$ e $a\divides b$, então $\gcd a b = a$.
\end{exercise}

\begin{exercise}
\topic{teoria de números}
	Se $ab\cong 0\pmod p$, onde $p$ é um primo, então $a\cong 0\pmod p$ ou $b\cong 0\pmod p$.
\end{exercise}

\begin{exercise}
\topic{teoria de números}
	$ax + by = c$ tem soluções sse $\gcd a b\divides c$.
\end{exercise}

\begin{definition}[Teorema de Fermat]
\label{tfermat}
\topic{teoria de números!teorema de Fermat}
	Seja $p$ um primo. Então
	$$
		a^{p - 1}\cong 1\pmod p
	$$
	para todo $a\ncong 0\pmod p$.
\end{definition}

\begin{definition}
\topic{teoria de números}
	$\tot n$ é o número de inteiros positivos menores do que $n$ que são coprimos com $n$.
\end{definition}

\begin{definition}[Teorema de Euler]
\label{teuler}
\topic{teoria de números!teorema de Euler}
	Se $a$ e $n$ são coprimos
	$$
		a^{\tot n}\cong 1\mod n.
	$$
\end{definition}

\begin{exercise}
\topic{teoria de números}
	Se $p$ é primo, encontre $\tot p$. Use isso para deduzir o \emph {Teorema de Fermat} (\ref{tfermat}) a partir do \emph {Teorema de Euler} (\ref{teuler}).
\end{exercise}

\begin{homework}
\topic{teoria de números}
	Prove que $\gcd a 0 = a$, se $a > 0$.
\end{homework}

\begin{homework}
\topic{teoria de números}
	Se $\gcd a c = 1$ e $c\divides ab$, então $c\divides b$.
\end{homework}
