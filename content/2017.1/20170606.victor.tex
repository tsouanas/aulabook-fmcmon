Nos exerc�cios abaixo, $G$ e $H$ sempre ser�o grupos.

\begin{exercise}
\topic{grupos!Homomorfismo}
Seja $f:G \to H$ um homomorfismo, $J$ um subgrupo de $H$ e defina $B$ como sendo a imagem inversa de $J$ por $f$. Prove que $B$ � subgrupo de $G$ e tamb�m que $Ker f$ � subconjunto de $B$.
\end{exercise}

\begin{exercise}
\topic{grupos!Homomorfismo}
Seja $f:G \to H$ homomorfismo e $m$ um inteiro relativamente primo com a ordem de $H$. Mostre que se $x^m$ for elemento de $Ker f$ ent�o $x$ � elemento de $Ker f$.
\end{exercise}

\begin{exercise}
\topic{grupos!Homomorfismo}
Seja $f:G \to H$ homomorfismo sobrejetor. Prove que se todo elemento de $G$ possui ordem finita, ent�o todo elemento de $H$ possui ordem finita.
\end{exercise}

\begin{exercise}
\topic{an�is}
Seja $A$ um anel. Prove que se o grupo aditivo de $A$ � c�clico, ent�o $A$ � um anel comutativo.
\end{exercise}
