Nos exercícios abaixo, $G$ e $H$ sempre serão grupos.

\begin{exercise}
\topic{grupos!homomorfismo}
Seja $f:G \to H$ um homomorfismo, $J$ um subgrupo de $H$ e defina $B$ como sendo a imagem inversa de $J$ por $f$.
Prove que $B$ é subgrupo de $G$ e também que $\ker f$ é subconjunto de $B$.
\end{exercise}

\begin{exercise}
\topic{grupos!homomorfismo}
Seja $f:G \to H$ homomorfismo e $m$ um inteiro relativamente primo com a ordem de $H$.
Mostre que se $x^m$ é elemento de $\ker f$ então $x$ é elemento de $\ker f$.
\end{exercise}

\begin{exercise}
\topic{grupos!homomorfismo}
Seja $f:G \to H$ homomorfismo sobrejetor. Prove que se todo elemento de $G$ possui ordem finita, então todo elemento de $H$ possui ordem finita.
\end{exercise}

\begin{exercise}
\topic{anéis}
Seja $A$ um anel.
Prove que se o grupo aditivo de $A$ é cíclico, então $A$ é um anel comutativo.
\end{exercise}
