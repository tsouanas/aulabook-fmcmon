\topic{teoria de números!divisão de Euclides}
\begin{definition}[Divisão de Euclides]
\label{ediv}
\topic{teoria de números!divisão de Euclides}%
    Dados inteiros $a$ e $b$ com $b > 0$, existem inteiros $q$ e $r$ tais que
    $$
        a = bq + r, \qquad 0 \leq r < b.
    $$
    Além disso, os $q$ e $r$ são determinados unicamente.
\end{definition}

\begin{exercise}
\topic{teoria de números!divisão de Euclides}%
    Prove a unicidade dos $q$ e $r$.
\end{exercise}

\begin{exercise}
\topic{teoria de números!divisão de Euclides}%
    Se $a, b\in\ints$ com $b > 0$, então $(a,b) = (b,r)$, onde $r$ é o resto da divisão de $a$ por $b$.
\end{exercise}

\topic{teoria de números!algoritmo de Euclides}[Algoritmo de Euclides]
\begin{definition}
\label{ealg}
\topic{teoria de números!algoritmo de Euclides}%
    Sejam $a, b$ inteiros positivos. Para encontrar o $\gcd a b$, aplicamos a \emph{Divisão de Euclides} (\ref{ediv}) repetidamente até chegar em resto 0. O $\gcd a b$ será igual ao último resto diferente de 0 que foi obtido.
    \begin{itemize}[--]
        \item Por que o \emph{Algoritmo de Euclides} sempre termina? 
    \end{itemize}
\end{definition}

\begin{exercise}
\label{egcd}
\topic{teoria de números!algoritmo de Euclides}%
    Usando o \emph{Algoritmo de Euclides} (\ref{ealg}), encontre o $\gcd {101} {73}$.
\end{exercise}

\begin{definition}[Algoritmo Estendido de Euclides]
\topic{teoria de números!algoritmo de Euclides!estendido}%
    O máximo divisor comum $\gcd a b$ de dois inteiros $a$ e $b$ pode ser escrito como uma combinação linear de $a$ e $b$. Usamos o \emph{Algoritmo Estendido de Euclides} para encontrar os inteiros $s, t\in\ints$ que satisfazem a
    $$
        \gcd a b = as + bt.
    $$
\end{definition}

\begin{exercise}
\topic{teoria de números!algoritmo de Euclides!estendido}%
    Continuando o \ref{egcd}, encontre $t,s\in\ints$ tais que
    $$
        \gcd {101} {73} = 101t + 73s.
    $$
\end{exercise}

\begin{homework}
\topic{teoria de números!algoritmo de Euclides}%
    Prove a corretude do \emph{Algoritmo de Euclides} (\ref{ealg}).
\end{homework}

