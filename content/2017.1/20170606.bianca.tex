\begin{exercise}
\topic{combinatória!enumerativa}
	No ``meio'' de uma ``cidade infinita'', tem um motorista no seu carro. Ele está parado numa interseção onde há $3$ opções: virar à esquerda; dirigir reto; virar à direita. Em seu tanque tem $a$ unidades de combustível, e sempre gasta $1$ para dirigir até a próxima interseção. De quantas maneiras diferentes ele pode dirigir até seu combustível acabar?
\end{exercise}

\begin{exercise}
\topic{combinatória!enumerativa}
	Aleco e Bego são dois sapos. Eles estão na frente de uma escada com $11$ degraus. No $6$º degrau, tem Cátia, uma cobra, com fome. Aleco pula $1$ ou $2$ degraus para cima. Bego, $1$, $2$ ou $3$. E ele é tóxico: se Cátia o comer, ela morre na hora.
	\begin{enumerate}[(1)]
		\item Por enquanto, Cátia está dormindo profundamente.
			\begin{enumerate}[(a)]
				\item De quantas maneiras Aleco pode subir a escada toda?
				\item De quantas maneiras Bego pode subir a escada toda?
			\end{enumerate}
		\item Cátia acordou!
			\begin{enumerate}[(a)]
				\item De quantas maneiras Aleco pode subir a escada toda?
				\item De quantas maneiras Bego pode subir a escada toda?
			\end{enumerate}
		\item Bego começou a subir a escada\ldots Qual é a probabilidade que Cátia morra? (Considere que antes de começar, ele já decidiu seus saltos e não tem percebido a existência da cobra.)
	\end{enumerate}
\end{exercise}

\begin{exercise}
\topic{combinatória!enumerativa}
	Prove que para todos os inteiros positivos $n$ e $r$ temos
	$$
		\comb n r = \comb {n - 1} r + \comb {n - 1} {r - 1}.
	$$
\end{exercise}

\begin{homework}
\topic{combinatória!enumerativa}
	Prove que
	$$
		\perm n n = \perm n {n-1}
	$$
	para todos os inteiros positivos $n$.
\end{homework}

\begin{homework}
\topic{combinatória!enumerativa}
	Prove que
	$$
		\perm n 1 + \perm m 1 = \perm {n + m} 1
	$$
	para todos os inteiros positivos $m$ e $n$.
\end{homework}
