Nos exercícios abaixo $G$ é grupo, $H$ e $K$ são subgrupos de $G$ e $p$ é primo.

\begin{exercise}
\topic{grupos!teorema de Lagrange}
    A ordem de $H \intersection K$ é um divisor comum da ordem de $H$ e da ordem de $K$.
\end{exercise}

\begin{exercise}
\topic{grupos!teorema de Lagrange}
    A ordem de $H \intersection K$ é um divisor comum da ordem de $H$ e da ordem de $K$.
    Se ordem de $H = m$, ordem de $K = n$ e $\gcd m n = 1$, mostrar que $H \intersection K = \{e\}$.
\end{exercise}

\begin{exercise}
\topic{grupos!teorema de Lagrange}
    A ordem de $H \intersection K$ é um divisor comum da ordem de $H$ e da ordem de $K$.
   Se $G$ possui um elemento de ordem $p$ e um elemento de ordem $q$, onde $p$ e $q$ são primos distintos, mostrar que a ordem de $G$ é múltiplo de $pq$.
\end{exercise}

\begin{exercise}
\topic{grupos!teorema de Lagrange}
    A ordem de $H \intersection K$ é um divisor comum da ordem de $H$ e da ordem de $K$.
    Se a ordem de $G$ é n, mostrar que $x^n = e$ para todo $x$ em $G$.
\end{exercise}

\begin{exercise}
\topic{grupos!teorema de Lagrange}
    A ordem de $H \intersection K$ é um divisor comum da ordem de $H$ e da ordem de $K$.
    $H \neq K$ e $H$ e $K$ possuem mesma ordem $p$, mostrar que que $H \intersection K = \{e\}$.
\end{exercise}
