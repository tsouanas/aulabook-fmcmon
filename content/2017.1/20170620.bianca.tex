\begin{definition}[Isomorfismo de ordem]
\topic{posets!isomorfismo de ordem}
	Dizemos que $P$ e $Q$ são isomorfos de ordem se existe um mapeamento bijetivo $\varphi$ de $P$ para $Q$ tal que $x \leq y$ em $P$ se e somente se $\varphi(x) \leq \varphi(y)$ em $Q$.
\end{definition}
	
\begin{definition}[A relação de cobertura]
\topic{posets!cobertura}
	Seja $P$ um poset e sejam $x, y \in P$. Dizemos que $x$ é coberto por $y$, e escrevemos $x \covby y$, se $x < y$ e $x \leq z < y$ implica $z = x$.
\end{definition}

\begin{exercise}
\topic{posets}
	Sejam P e Q posets finitos e seja ${\varphi}:P \to Q$ um mapeamento bijetivo. Então, os seguintes são equivalentes:
	\begin{enumerate}[(i)]
		\item $\varphi$ é um isomorfismo de ordem;
		\item $x < y$ em $P$ se e somente se $\varphi(x) < \varphi(y)$ em $Q$;
		\item $x \covby y$ em $P$ se e somente se $\varphi(x) \covby \varphi(y)$.
	\end{enumerate}
\end{exercise}

\begin{exercise}
\topic{posets}
	Dois posets finitos $P$ e $Q$ são isomorfos de ordem se e somente se podem ser desenhados com diagramas idênticos.
\end{exercise}

\begin{exercise}
\topic{posets}
	Existe uma lista de $16$ diagramas de posets de quatro elementos, tal que todo poset de quatro elementos pode ser representado por um dos diagramas nessa lista. Encontre essa lista.
\end{exercise}
