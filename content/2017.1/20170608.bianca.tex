\begin{exercise}
\topic{combinatória enumerativa}
	De quantas maneiras $10$ pessoas podem sentar numa fila de forma que certas duas pessoas não fiquem juntas?
\end{exercise}

\begin{exercise}
\topic{combinatória enumerativa}
	Quantos inteiros entre $1000$ e $9999$ incluso têm dígitos distintos? Quantos deles são ímpares?
\end{exercise}

\begin{exercise}
\topic{combinatória enumerativa}
	Com os dígitos $0, 1, 2, 3, 4, 5, 6$, quantos números de quatro dígitos (distintos) podem ser construídos? Quantos deles são pares?
\end{exercise}

\begin{exercise}
\topic{combinatória enumerativa}
	Quantos inteiros maiores que $53000$ têm as seguintes propriedades:
	\begin{enumerate}[(a)]
		\item Os itens do inteiros são distintos;
		\item Os dígitos $0$ e $9$ não ocorrem no número?
	\end{enumerate}
\end{exercise}

\begin{homework}
\topic{combinatória enumerativa}
	De quantas maneiras é possível $8$ pessoas sentarem numa mesa redonda?
\end{homework}

\begin{homework}
\topic{combinatória enumerativa}
	Na pergunta anterior, qual seria a resposta se certas duas das oito pessoas não podem sentar em lugares adjacentes?
\end{homework}

\begin{homework}
\topic{combinatória enumerativa}
	Com os dígitos $1,2,3,4,5$, quantos números de quatro dígitos (distintos) podem ser construídos? Quantos deles são ímpares?	
\end{homework}

\begin{homework}
\topic{combinatória enumerativa}
	Quantos cubos diferentes cujas faces sejam numeradas de $1$ a $6$ podem ser feitos se a soma dos números de cada par de faces opostas é $7$?
\end{homework}
