\begin{exercise}
\topic{grupos!teorema de Lagrange}
	Se $G$ tem ordem $n$, então $x^n = e$ para todo $x$ em $G$.
\end{exercise}

\begin{exercise}
\topic{grupos!teorema de Lagrange}
	Seja $G$ de ordem $pq$, onde $p$ e $q$ são primos. Ou $G$ é cíclico, ou todo elemento $x \neq e$ em $G$ tem ordem $p$ ou $q$.
\end{exercise}

\begin{exercise}
\topic{grupos!homomorfismo}
	Se $f:G \to H$ é um homomorfismo com kernel $K$, então $f$ é injetiva sse $K = \set e$.
\end{exercise}
