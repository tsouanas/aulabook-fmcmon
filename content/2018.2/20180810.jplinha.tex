\begin{exercise}
\topic{lógica!proposicional}
    Traduza as seguintes frases para fórmulas de lógicas.
    \begin{enumerate}[a)]
        \item João foi ao supermercado, ou ficaremos sem ovos.
        \item Joel vai sair de casa e não voltará.
    \end{enumerate}
\end{exercise}

\begin{exercise}
\topic{lógica!proposicional}
    Traduza para frases em português as seguintas fórmulas de lógica.
    \begin{enumerate}[a)]
        \item $(\lnot s \land l) \lor s$, onde $s \leteq$ "Giordano é burro" e 
            $l \leteq$ "Giordano é preguiçoso".
        \item $\lnot s \land (l \lor s)$, onde $s$ e $l$ tem a mesma definição.
        \item $\lnot(s \land l) \lor s$, onde $s$ e $l$ tem a mesma definição.
    \end{enumerate}
\end{exercise}

\begin{exercise}
\topic{lógica!proposicional}
    Analise as fórmulas lógicas nas seguintes afirmações.
    \begin{enumerate}[a)]
        \item Vamos ter um livro para ler ou tarefa para casa, mas não vamos ter
            tarefas para casa e uma prova.
        \item Você não vai esquiar, ou você vai e não terá neve.
        \item $\sqrt{7} \nleq 2$
    \end{enumerate}
\end{exercise}

\begin{exercise}
\topic{lógica!proposicional}
    Mostre que
    \begin{enumerate}[(i)]
        \item $p \land (p \lor q) \iff p$
        \item $p \lor (p \land q) \iff p$
    \end{enumerate}
\end{exercise}
