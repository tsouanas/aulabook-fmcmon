\begin{exercise}
\topic{conjuntos!teoria axiomática}
Considere o axioma seguinte:

\begin{description}
	\item[ZF3*] $\forall a \forall b \forall c \left( (a \ne b \land b \ne c \land c \ne a) \rightarrow \exists s \forall x(x \in s \leftrightarrow x=a \lor x=b \lor x=c)  \right)$
\end{description}

\begin{enumerate}[a)]
\item No sistema ZF1+ZF2+ZF3+ZF4+ZF5+ZF6, construa um conjunto de cardinalidade 3.
\item Prove que o ZF3 pode ser substituído pelo ZF3* ``sem perder nada''; isto é, prove, no sistema ZF1+ZF2+ZF3*+ZF4+ZF5+ZF6, que se $a$ e $b$ são conjuntos então $\{a,b\}$ também é.
\item O resultado do item anterior permanece válido quando se remove o ZF5 do sistema?
\end{enumerate}
\end{exercise}

\begin{exercise}
\topic{conjuntos!teoria axiomática}
Considere o axioma seguinte:

\begin{description}
	\item[CONS] $\forall h \forall t \exists s \forall x (x \in s \leftrightarrow x=h \lor x \in t)$
\end{description}

\begin{enumerate}[a)]
\item No sistema ZF1+ZF2+CONS, prove o ZF3 como teorema.
\item Mostre que é impossível provar o CONS no sistema ZF1+ZF2+ZF3. 
\item No sistema ZF1+ZF2+ZF3+ZF4+ZF5+ZF6, prove o CONS como teorema.
\end{enumerate}
\end{exercise}