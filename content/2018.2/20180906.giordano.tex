\begin{exercise}
\topic{teoria de números}%
Prove que para todos $a, b, c \in \nats$, se $a \divides b$ e $a \divides c$, então $a \divides bx + cy$ para quaisquer $x, y \in \ints$
\end{exercise}

\begin{exercise}
\topic{teoria de números}%
Prove que para todo $n \in\nats$, $3 \ndivides 2n + 3$
\end{exercise}

\begin{exercise}
\topic{teoria de números}%
Prove que para todos $a, b, c \in\nats$, se $a \divides b$ e $a \ndivides c$, então $a \ndivides b + c$
\end{exercise}

\begin{exercise}
\topic{funções}%
Sejam $f : A \to B$ e $g : B \to C$, prove que se $f$ e $g$ são injetivas, então $g \fcompose f$ é injetiva.
\end{exercise}

\begin{exercise}
\topic{funções}%
Sejam $f : A \to B$ e $g : B \to C$, prove que se $f$ e $g$ são sobrejetivas, então $g \fcompose f$ é sobrejetiva.
\end{exercise}

\begin{exercise}
\topic{funções}%
Defina uma função $f : \nats\to\ints$ que seja bijetiva, isto é, prove que sua função é realmente uma bijeção.
\end{exercise}

