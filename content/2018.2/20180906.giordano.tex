\begin{exercise}
\topic{teoria de números}%
Prove que para todos $a, b, c \in \nats$, se $a \divides b$ e $a \divides c$, então $a \divides bx + cy$ para quaisquer $x, y \in \ints$
\end{exercise}

\begin{exercise}
\topic{teoria de números}%
Prove que para todo $n \in\nats$, vale $3 \ndivides 2n + 3$
\end{exercise}

\begin{exercise}
\topic{teoria de números}%
Sejam $a, b, c \in\nats$, prove que
	$$ a \divides b \mland a \ndivides c \implies a \ndivides b + c $$
\end{exercise}

\begin{exercise}
\topic{funções}%
Sejam $f : A \to B$ e $g : B \to C$, prove que
	\begin{enumerate}[(i)]
		\item $ f, g \text{ injetivas} \implies g \fcompose f \text{ injetiva}$

		\item $f, g \text{ sobrejetivas} \implies g \fcompose f \text{ sobrejetiva}$
	\end{enumerate}
\end{exercise}

\begin{exercise}
\topic{funções}%
Defina uma $f : \nats\to\ints$ função bijetiva, isto é, prove que sua definição realmente define uma função, e que é bijetiva.
\end{exercise}

