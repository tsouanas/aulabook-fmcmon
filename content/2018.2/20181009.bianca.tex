\begin{exercise}
\topic{relações!composição}
Seja $R$ uma preordem num conjunto $A$. Prove que $R$ é idempotente, ou seja, $R = R \rcompose R$.
\end{exercise}

\begin{exercise}
\topic{relações!composição}
Seja $S$ uma relação binária no $\reals$ tal que

$$
    (S \rcompose S) \text{ é irreflexiva}.
$$

Qual é o gráfico da $S$? Prove tua resposta.
\end{exercise}

\begin{exercise}
\topic{relações!equivalência}
Defina com texto completo o conjunto quociente.
\end{exercise} 

\begin{exercise}
\topic{relações!equivalência}
Defina com texto completo o que é uma partição.
\end{exercise} 

\begin{exercise}
\topic{relações!equivalência}
Seja $\sim$ uma relação de equivalência num conjunto $A$. Prove que o conjunto quociente $A/\sim$ é uma partição de $A$. 
\end{exercise} 
