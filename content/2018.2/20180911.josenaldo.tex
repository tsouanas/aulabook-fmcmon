\begin{exercise}
\topic{funções!injetividade}
\topic{funções!pré-imagem}
\label{inj-implies-less-than-two}
Seja $f : A \to B$ injetiva. Prove que, para todo $Y \subseteq B$ unitário, o conjunto $f^{-1}[Y]$ tem menos de dois elementos.
\end{exercise}

\begin{homework}
\topic{funções!sobrejetividade}
\topic{funções!pré-imagem}
\topic{funções!bijetividade}
No Exercício \ref{inj-implies-less-than-two}, prove que, se $f$ é sobrejetiva, então $f^{-1}[Y] \ne \emptyset$. Conclua que $f$ ser bijetiva e $Y$ ser unitário implicam que $f^{-1}[Y]$ é unitário.
\end{homework}