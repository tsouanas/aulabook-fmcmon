\begin{exercise}
\topic{funções}
    Sejam $f: A \to B,$ $ h: B \to A$. Suponha que $f$ tem inversa. Mostre que $f$ satisfaz as duas leis de cancelamento.
    \begin {enumerate}[(a)]
        \item Se $f \fcompose h = f \fcompose k$, então $h = k$
        \item Se $h \fcompose f = k \fcompose f$, então $h = k$
    \end {enumerate}
\end{exercise}

\begin{exercise}
\topic{funções}
    Sejam $A \not = \emptyset$ um conjunto e $f : A \to \mathcal P A$ definida pela equação:
    $$
        f(a) = \set a
    $$
    \begin {enumerate} [(a)]
        \item $f$ é injetora?
        \item $f$ é sobrejetora?
    \end {enumerate}
\end{exercise}

\begin{exercise}
\topic{funções}
    Mostre que se $f$ tem uma inversa, então ela é única.
\end{exercise}
