\begin{exercise}
\topic{teoria de números!divisibilidade}%
	Sejam $a,b,c \in \nats$, demonstre que
	$$a \divides b \mland a \divides c \implies a \divides bx + cy \text{ para quaisquer } x,y \in \ints$$
\end{exercise}

\begin{definition}[Máximo Divisor Comum]
\topic{teoria de números!mdc}%
	Considere $a$, $b$ e $d\in \ints$, defina $mdc(a,b) = d$ se, e somente se:
	\begin{enumerate}[(i)]
		\item $d \geq 0$ (não negativo)
		\item $d \divides a \mland d \divides b$ (divisor comum)
		\item $(\forall n \in \nats)[ (n \divides a \mland n \divides b) \implies n \divides d ]$ (maior em relação à divisibilidade)
	\end{enumerate}
\end{definition}

\begin{exercise}
\topic{teoria de números!mdc}%
	Seja $a \in \nats$, demonstre que valem os seguintes itens:
	\begin{enumerate}[(a)]
		\item $mdc(a,a) = a$
		\item $mdc(0,a) = a$
	\end{enumerate}
\end{exercise}
