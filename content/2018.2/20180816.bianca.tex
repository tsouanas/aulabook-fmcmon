\begin{exercise}
\topic{teoria de números!divisibilidade}
    Prove que a relação $\divs$ é uma ordem parcial no $\nats$.
\end{exercise}

\begin{exercise}
\topic{teoria de números!divisibilidade}
    Prove que para todo $n \in \ints$, se $3\not{\divs} \ n$, então $3 \divs n^2 - 1$.
\end{exercise}

\begin{exercise}
\topic{indução}
    Os nũmeros Fibonacci são definidos recursivamente assim:
    $$
        \begin{aligned}
            F_0 &= 0\\
            F_1 &= 1\\
            F_{n+2} &= F_{n+1} + F_n
        \end{aligned}
    $$
    Prove que para todo $n \in \nats$,
    $$
       \sum_{i=0}^{n} F_i = F_{n+2} - 1.
    $$
\end{exercise}

\begin{homework}
\topic{congruência}
	Prove que $\bullet \cong \bullet \pmod m$ é uma relação de equivalência nos inteiros.
\end{homework}
