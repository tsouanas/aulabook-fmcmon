\begin{exercise}
    \topic{relações!equivalência}%
    Defina no $\ints \times \ints_{\neq 0}$ a seguinte relação
    $$
    \tup{a,b} \approx \tup{c,d} \iffdf ad = bc
    $$
    mostre que essa relação é de equivalência e descreva suas classes de
    equivalência e conjunto quociente.
\end{exercise}

\begin{exercise}
    \topic{relações!equivalência}%
    \topic{relações!ordem}%
    \topic{relações!pré-ordem}%
    Dadas as seguintes relações no $(\ints \to \ints)$
    \begin{align}
      f \sim g    &\iffdf
                  (\exists u \in \ints)(\forall x \in \ints)[f(x) = g(x + u)] \\
      f \approx g &\iffdf
                  (\exists u \in \nats)(\forall x \in \ints)[f(x) = g(x + u)] \\
      f \wr g     &\iffdf
                  (\exists u \in \ints)(\forall x \in \ints)[f(x) = g(x) + u] \\
      f \wr\!\wr\; g &\iffdf
                  (\exists u \in \nats)(\forall x \in \ints)[f(x) = g(x) + u]
    \end{align}
    decida se cada uma é (ir)reflexiva, transitiva ou (a(nti)s)simétrica.
\end{exercise}

\begin{homework}
    \topic{relações}%
    Sabendo que a $(2)$ do exercício anterior não é nem simétrica nem
    antissimétrica, construa contraexemplos que mostrem essa afirmação.
\end{homework}