\begin{definition}[Naturais]
    \topic{naturais}%
    \topic{bnf}%
    \topic{recursão}%
    Definimos o conjunto dos naturais recursivamente, usando a notação BNF, da
    seguinte forma:

    $$ \tup{\textbf{Nat}}\ ::=\ 0\ \vert\ S\ \tup{\textbf{Nat}} $$

\end{definition}

\begin{exercise}
    \topic{naturais}%
    \topic{recursão}%
    Defina a operação de adição sobre o $ \tup{\textbf{Nat}} $ definido acima.
\end{exercise}

\begin{exercise}
    \topic{naturais}%
    \topic{indução}%
    Mostre que a adição sobre o $ \tup{\textbf{Nat}} $ do exercício anterior é
    associativa. Ou seja, para todos $a, b, c$ naturais, $a + (b + c) = (a + b)
    + c$.
\end{exercise}

\begin{homework}
    \topic{naturais}%
    \topic{indução}%
    Mostre que a adição sobre o $ \tup{\textbf{Nat}} $ é comutativa. Ou seja,
    para todos $m, n$ naturais, $m + n = n + m$.
\end{homework}

\begin{homework}
    \topic{naturais}%
    \topic{indução}%
    Ainda sobre o $ \tup{\textbf{Nat}} $, defina a operação de multiplicação e
    tente provar sua distributividade, associatividade e comutatividade.
\end{homework}
