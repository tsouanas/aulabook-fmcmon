\begin{exercise}
    \topic{teoria de números!congruência}%
    Sejam $m, a, b \in \ints$. Prove que, se $a \congmod m b$, então para todo
    $x \in \ints$,
    \begin{enumerate}[(i)]
        \item $a + x \congmod m {b + x}$
        \item $ax \congmod m {bx}$
        \item $-a \congmod m {-b}$
    \end{enumerate}
\end{exercise}

\begin{exercise}
    \topic{teoria de números!congruência}%
    Sejam $m, a, k \in \ints$ e seja $c \in \ints$ tal que $\gcd c m = 1$.
    Prove que
    $$
    ca \congmod m {cb} \implies a \congmod m b
    $$
\end{exercise}

\begin{exercise}
    \topic{grupos!ordem}%
    Sejam $G$ grupo, $a \in G$ e $m \in \ints$. Então $a^m = e \iff o(a)\divides
    m$.
\end{exercise}

\begin{exercise}
    \topic{grupos!ordem}%
    Sejam $G$ grupo e $a \in G$. Se $o(a) = km$, então $o(a^k) = m$.
\end{exercise}

\begin{exercise}
    \topic{grupos!subgrupos}%
    Seja $G$ grupo. Prove que para todo $a \in G$, $\left<a\right> \subgroup G$.
\end{exercise}
