\begin{exercise}
\topic{grupos!ordem}
	Sejam $G$ grupo e $a \in G$, e suponha $o(a) = n \in \nats$. Existem exatamente $n$ potências diferentes de $a$.
\end{exercise}

Sejam $a, b$ elementos de um grupo $G$. Prove o seguinte:

\begin{exercise}
\topic{grupos!ordem}
	$o(a) = 1$ sse $a = e$.
\end{exercise}

\begin{exercise}
\topic{grupos!ordem}
	Se $o(a) = n$, então $a^{n - r} = (a^r)^{-1}$.	
\end{exercise}

\begin{exercise}
\topic{grupos!ordem}
	$o(a) = o(bab^{-1})$.
\end{exercise}

\begin{exercise}
\topic{grupos!ordem}
	A ordem de $ab$ é a mesma que a ordem de $ba$.	
\end{exercise}
