\begin{definition}[Inverso módulo $m$]% para dor nomes em teoremas/definições bota em []
    Sejam $a,a',m\in\ints$. Chamamos $a'$ o inverso (multiplicativo) de $a$ módulo $m$ sse
    $$
        aa'\cong 1\pmod m.
    $$
    \begin{itemize}[--]
        \item O inverso é único;
        \item O $a$ tem inverso módulo $m$ sse $\gcd a b = 1.$
    \end{itemize}
\end{definition}

\begin{exercise}
\topic{congruência}
    Prove a unicidade do inverso módulo $m$. 
\end{exercise}

\begin{exercise}
\topic{congruence!chinese remainder theorem}
    Usando o Teorema Chinês do Resto, ache todos os inteiros $x\in\ints$ que satisfazem o sistema de congruências:
    \begin{align*}
        x&\cong 2\pmod 9\\
        x&\cong 1\pmod 5\\
        x&\cong 2\pmod 4
    \end{align*}
\end{exercise}

\begin{exercise}
\topic{congruência!teorema!Chinês do resto}
    Ache todos os inteiros $x\in\ints$ com $\abs x < 64$ que satisfazem o sistema de congruências:
    \begin{align*}
        x &\cong 1\pmod 3\\
        3x&\cong 1\pmod 4\\
        4x&\cong 2\pmod 5
    \end{align*}
\end{exercise}

\begin{homework}
\topic{congruência!teorema!Chinês do resto}
    Use o Teorema Chinês do Resto para resolver os sistemas de congruências:
    % vou mudar isso tambem, principalmente para ter como referencia
    $$
    \text{(a)}
    \left\{
            \begin{aligned}
                x&\cong 1\pmod 3\\
                x&\cong 2\pmod 5\\
                x&\cong 3\pmod 7
            \end{aligned}
    \right.
    \qquad
    \underbrace{
    \underbrace{
            \begin{aligned} % o align* abre sozinho math-mode
                x&\cong 2\pmod 4\\
                x&\cong 3\pmod 5\\
                x&\cong 4\pmod 9\\
                x&\cong 5\pmod {13}
            \end{aligned}
        }_{\text{(b)}}
    \qquad
    \overbrace{
            \begin{aligned}
                x&\cong 3\pmod 4\\
               5x&\cong 1\pmod 7\\
                x&\cong 2\pmod 9
            \end{aligned}
        }^{\text{(c)}} % (como nao deu erro sem o \text, ele aceitou o \pmod, ou seja, esta em math-mode by default, por isso o \text{} aqui)
    }_{\text{dica: comeca com essas}}
    $$
    % e mais um jeito
    % veja os dois e escolhe um
    % mais um tip: o \phantom{..} creates a "phantom" text (invisible)
    % it's useful here to align the pmods that are on colums with two digits
    % like the second column here...
    % if i place 04, 05, 09, etc. it's ugly as fuck, so i'll put the 0 in a \phantom
    \begin{align*}
        x&\cong 1\pmod 3 &    x&\cong 2\pmod {\phantom04}  &   x&\cong 3\pmod 4\\
        x&\cong 2\pmod 5 &    x&\cong 3\pmod {\phantom05}  &  5x&\cong 1\pmod 7\\
        x&\cong 3\pmod 7 &    x&\cong 4\pmod {\phantom09}  &   x&\cong 2\pmod 9\\
         &               &    x&\cong 5\pmod {13} 
    \end{align*}
    % vamo compilar
\end{homework}

\begin{homework}
    Encontre o menor inteiro positivo que deixa os restos $8$, $7$ e $11$ quando dividido por $7$, $11$ e $15$, respectivamente.
\end{homework}
