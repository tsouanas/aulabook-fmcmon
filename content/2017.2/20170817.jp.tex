\begin{exercise}
\topic{conjuntos!intensão/extensão}
Determine quais dos seguintes conjuntos são definidos intensionalmente vs extensionalmente. \\
(a) $\set{ 0, 1, 2, 3, 4, 5, 6, 7, 8, 9 }$ \\
(b) $\setstt {x \in \ints} {$x$ é par}$ \\
(c) $\setstt { \text{$x$ par} } {$x = p + q$, onde $p, q$ são primos}$
\end{exercise}

\begin{exercise}
\topic{conjuntos!álgebra booleana}
Demonstre que, dados $A, B, C \subseteq U$, $A \inter (B \union C) = (A \inter B) \union (A \inter C)$.
\end{exercise}
