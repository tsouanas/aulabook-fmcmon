\begin{exercise}
	\topic{grupos!ordem}
	\topic{grupos!geradores}
	Suponha que $G$ tem ordem $pq$, onde $p$ e $q$ são primos.
	Mostre que $G$ é cíclico ou todo elemento de $G$ tem ordem $p$ ou $q$.
\end{exercise}

\begin{exercise}
	\topic{grupos!ordem}
	\topic{grupos!geradores}
	Seja $G$ um grupo de ordem $4$.
	Mostre que $G$ é cíclico ou todo elemento de $G$ é o próprio inverso.
	Conclua que $G$ é abeliano.
\end{exercise}

\begin{exercise}
	\topic{grupos!ordem}
	Suponha que $G$ tem um elemento de ordem $p$ e um elemento de ordem $q$, onde $p$ e $q$ são primos distintos. Mostre que a ordem de $G$ é múltiplo de $pq$.
\end{exercise}

\begin{exercise}
	\topic{grupos!ordem}
	Suponha que $G$ tem um elemento de ordem $k$ e um elemento de ordem $n$. Mostre que a ordem de $G$ é múltiplo de $mmc(k, n)$.
\end{exercise}

\begin{exercise}
	\topic{grupos!ordem}
	Seja $p$ um primo. Prove que, em qualquer grupo finito, o número de elementos de ordem $p$ é múltiplo de $p-1$.
\end{exercise}
