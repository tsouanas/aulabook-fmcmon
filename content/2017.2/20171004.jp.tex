Seja $G$ um grupo, $a$ um elemento de $G$.
\begin{definition}[Teorema]
	\label{cyclic_modular}
	Para todo $n, m \in \ints$, $a^m = a^n$ se e somente se $m \cong n \pmod {o(a)}$.
\end{definition}

\begin{exercise}
	\topic{grupos!ordem}
	Prove o teorema \ref{cyclic_modular}.
\end{exercise}

Para os exercícios abaixo, assuma que as variáveis não-introduzidas são inteiros quaisquer.

\begin{exercise}
	\topic{grupos!ordem}
	Prove que $o(a^k) \divides o(a)$.
\end{exercise}

\begin{exercise}
	\topic{grupos!ordem}
	Seja $p$ um primo.
	Prove que se $a \not = e$ e $a^p = e$, então $o(a) = p$. 
\end{exercise}

\begin{exercise}
	\topic{grupos!ordem}
	Seja $n$ um número ímpar. 
	Prove que se $o(a) = n$, então $o(a^2) = n$.
\end{exercise}

\begin{exercise}
	\topic{grupos!ordem}
	Prove que se $o(a) = km$, então $o(a^k) = m$.
\end{exercise}

\begin{exercise}
	\topic{grupos!ordem}
	Prove que se $o(a) = km$ e $a^{rk} = e$, então $m \divides r$.
\end{exercise}

\begin{exercise}
	\topic{grupos!ordem}
	Prove que se $m \ndivides o(a)$, então $m \ndivides o(a^k)$.
\end{exercise}

