\begin{exercise}
\topic{conjuntos!teoria axiomática}
	Sem usar o (ZF3), mostre que para qualquer conjunto $a$ existe um conjunto que tem $a$ como seu único elemento.
\end{exercise}

\begin{exercise}
\topic{conjuntos!teoria axiomática}
	Considere a seguinte versão mais fraca do (ZF3):
	\newline
	\newline Dados quaisquer conjuntos $a$ e $b$, existe um conjunto $u$ tal que $a \in u$ e $b \in u$.
	\newline
	\newline Chame-a de (ZF3$'$). Prove que é possível substituir o (ZF3) pelo (ZF3$'$) sem perder nada. Ou seja, dados objetos $a, b$, mostre que existe o conjunto $\set {a, b}$ que consiste exatamente nesses objetos.
\end{exercise}

\begin{exercise}
\topic{conjuntos!teoria axiomática}
	Dado um conjunto $a$, justifique (usando os axiomas ZF) a existência do conjunto $\set {\set {x} \st x \in a}$.
\end{exercise}

\begin{exercise}
\topic{posets}
	Para $n \in \nats$, definimos o poset $\mathcal{D}_n \eqdf \sset {D_n} \divides$ onde $D_n \eqdf \set {d \in \nats \st d \divides n}$.
	\begin{enumerate}[(i)]
		\item Desenhe o diagrama Hasse de $\mathcal{D}_{30}$.
		\item Ache conjunto $A$ tal que $\mathcal{D}_{30} \knuthcong \sset {\powerset A} \subseteq$, e defina um isomorfismo $\varphi : D_{30} \to \powerset A$.
		\item Existe conjunto $B$ tal que $\mathcal{D}_0 \knuthcong \sset {\powerset B} \subseteq$? Se sim, ache o $B$ e defina um isomorfismo $\varphi : D_0 \to \powerset B$. Se não, prove que é impossível.
		\item Verdadeiro ou falso? $\mathcal{D}_0 \knuthcong \sset {\set {D_n \st n \in \nats}} \subseteq$.
	\end{enumerate}
\end{exercise}
