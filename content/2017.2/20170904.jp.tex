\begin{definition}[Lema 1: multiplicação respeita ordem decrescente]
	\label{multrespgt}
	Para todo $m$ e $n$ inteiros não-nulos, se $m > 0$ e $n > 0$, então $m \ntimes n > 0$.
\end{definition}

\begin{definition}[Lema 2: cancelamento da multiplicação respeita ordem decrescente]
	\label{multcancrespgt}
	Para todo $m$ e $n$ inteiros não-nulos, se $m > 0$ e $m \ntimes n > 0$, então $n > 0$.
\end{definition}

\begin{exercise}
\topic{relações!equivalência}
Seja $R$ uma relação binária nos inteiros não-nulos dada por $R(u, v)$ sse $u \ntimes v > 0$.
Mostre que $R$ é uma relação de equivalência (ou seja, que é reflexiva, simétrica e transitiva).
(Você pode usar os lemas acima).
\end{exercise}

\begin{homework}
Dada a definição $a > b$ sse existe $k$ positivo tal que $a = b + k$, prove os lemas \ref{multrespgt} e \ref{multcancrespgt}
\end{homework}