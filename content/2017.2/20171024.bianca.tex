\begin{exercise}
\topic{grupos!normais}
	Se $H \subgroup G$ de índice $2$, então $H \normal G$.
\end{exercise}

\begin{exercise}
\topic{grupos!normais}
	Se $H \subgroup G$ e $N \normal G$, então $H \intersection N \normal H$. 
\end{exercise}

\begin{exercise}
\topic{teoria de números!divisibilidade}
	Se $\gcd a c = 1$ e $c \divides ab$, então $c \divides b$.
\end{exercise}

\begin{exercise}
\topic{teoria de números!divisibilidade}
	Se $a \divides m$, $b \divides m$ e $\gcd a b = 1$, então $ab \divides m$.
\end{exercise}

\begin{exercise}
\topic{teoria de números!divisibilidade}
	Prove que $ax + by = c$ tem soluções sse $\gcd a b \divides c$.
\end{exercise}

\begin{exercise}
\topic{teoria de números!divisibilidade}
	Se $a > 0$ e $a \divides b$, então $\gcd a b = a$.
\end{exercise}

\begin{exercise}
\topic{teoria de números!divisibilidade}
	Se $ab \cong 0 \pmod p$, onde $p$ é primo, então $a \cong 0 \pmod p$ ou $b \cong 0 \pmod p$.
\end{exercise}
