\begin{definition}[Adição e Multiplicação]
	A função de adição nos números naturais é definida pela recursão
	$$
		\begin{aligned}
			n + 0 &= n, && \text{(a1)}\\
			n + (Sm) &= S(n + m). && \text{(a2)}
		\end{aligned}
	$$
	e a multiplicação é definida em seguida, usando adição, pela recursão
	$$
		\begin{aligned}
			n \cdot 0 &= 0, && \text{(m1)}\\
			n \cdot Sm &= (n \cdot m) + n. && \text{(m2)}
		\end{aligned}
	$$
\end{definition}

\begin{exercise}
\topic{naturais!Peano}
	Prove que adição é associativa, ou seja, satisfaz a identidade
	$$
		(n + m) + k = n + (m + k).
	$$
\end{exercise}

\begin{exercise}
\topic{naturais!Peano}
	Mostre que, para todo natural $n$, $0 + n = n$.
\end{exercise}

\begin{exercise}
\topic{naturais!Peano}
	Prove que, para todos $n, m$, $n + Sm = Sn + m$.
\end{exercise}

\begin{exercise}
\topic{naturais!Peano}
	Prove que adição é comutativa, ou seja, satisfaz a identidade
	$$
		n + m = m + n.
	$$
\end{exercise}
