\begin{exercise}
\topic{grupos!subgrupos}
	Seja $G$ grupo e $\emptyset \neq H \subseteq G$. Prove que
	$$
		H \subgroup G \iff (\forall a,b \in H)[ab^{-1} \in H].
	$$
\end{exercise}

\begin{exercise}
\topic{grupos!subgrupos}
	Mostre que $\subgroup$ é uma relação de ordem:
	\begin{enumerate}[(i)]
		\item $G \subgroup G$.
		\item $K \subgroup H \ \&\ H \subgroup K \implies K \subgroup G$.
		\item $H \subgroup G \ \&\ G \subgroup H \implies H = G$.
	\end{enumerate}
\end{exercise}

\begin{exercise}
\topic{grupos!ordem}
	Prove que a ordem de $a^{-1}$ é a mesma que a ordem de $a$.
\end{exercise}

\begin{exercise}
\topic{grupos!ordem}
	Sejam $G$ grupo, $a \in  G$ e $m \in \ints$. Prove que
	$$
		a^{m} = e \iff o(a) \divides m. 
	$$
\end{exercise}

\begin{exercise}
\topic{grupos!ordem}
	Se $a^p = e$ onde $p$ é um número primo, então $a$ tem ordem $p$ ($a \neq e$).
\end{exercise}

\begin{exercise}
\topic{grupos!ordem}
	A ordem de $a^k$ é um divisor (fator) da ordem de $a$.
\end{exercise}
