\begin{exercise}
\label{lists-def}
\topic{listas}
\topic{bnf}
Defina o conjunto $\mathcal{L}_A$ das listas de elementos de um tipo A usando BNF.
\end{exercise}

\begin{exercise}
\topic{listas}
\topic{recursão}
Defina a operação $append: \mathcal{L}_A \times \mathcal{L}_A \to \mathcal{L}_A$ de concatenação de duas listas. 
\end{exercise}

\begin{exercise}
\topic{provas!indução}
Demonstre que a $append$ definida no exercício anterior é associativa.
\end{exercise}

\begin{homework}
\topic{listas}
\topic{recursão}
Defina a operação $reverse: \mathcal{L}_A \to \mathcal{L}_A$, que faz a inversão de uma lista. Prove que $reverse$ é involutiva, ou seja, que $reverse \circ reverse = id$.
\end{homework}
