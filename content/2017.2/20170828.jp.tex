\def\List#1{{\mathcal L}_{#1}}
\def\append{{\mathit {append}}}
\def\reverse{{\mathit {reverse}}}
\def\id{{\mathit {id}}}

\begin{exercise}
\label{lists-def}
\topic{listas}
\topic{bnf}
Defina o conjunto $\List A$ das listas de elementos de um tipo $A$ usando BNF.
\end{exercise}

\begin{exercise}
\topic{listas}
\topic{recursão}
Defina a operação $\append: \List A \times \List A \to \List A$ de concatenação de duas listas.
\end{exercise}

\begin{exercise}
\topic{provas!indução}
Demonstre que a $\append$ definida no exercício anterior é associativa.
\end{exercise}

\begin{homework}
\topic{listas}
\topic{recursão}
Defina a operação $\reverse: \List A \to \List A$, que faz a inversão de uma lista.
Prove que $\reverse$ é involutiva, ou seja, que $\reverse\of\reverse = \id$.
\end{homework}
