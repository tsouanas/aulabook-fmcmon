\begin{exercise}
\topic{naturais!Peano}
\topic{bnf}
Defina o conjunto dos números naturais usando BNF.
\end{exercise}

\begin{exercise}
\topic{naturais!Peano}
\topic{recursão}
Defina a operação de adição sobre os números naturais como definidos acima.
\end{exercise}

\begin{exercise}
\topic{indução}
Demonstre que a adição definida no exercício anterior é associativa.
Isso é, mostre que para todo $m, n, p$ naturais, $m + (n + p) = (m + n) + p$.
\end{exercise}

\begin{definition}[Lema 1 da adição]
	\label{lema1}
	Para todo $n$ natural, $n + 0 = n$.
\end{definition}

\begin{definition}[Lema 2 da adição]
	\label{lema2}
	Para todo $m, n$ naturais, $S(m + n) = m + Sn$.
\end{definition}

\begin{homework}
\topic{indução}
Demonstre \ref{lema1} (p.~\pageref{lema1}) e \ref{lema2} (p.~\pageref{lema2}).
\end{homework}

\begin{exercise}
\topic{indução}
\topic{lemmas}
Demonstre que a adição é comutativa (ou seja, que para todo $m, n$ naturais, $m + n = n + m$).
Use coisas que demonstramos anteriormente.
\end{exercise}


