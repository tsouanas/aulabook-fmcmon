\begin{definition}
	Denotamos por $S_n$ o conjunto das permutações em $n$ elementos, ou, equivalentemente, o conjunto das funções bijetoras sobre $\{0, 1, \dots, n\}$.
	Usaremos a notação
	$\paren{
		\begin{array}{cccc}
			       1  &        2  & \dots &        n \\
			\sigma(1) & \sigma(2) & \dots & \sigma(n)
		\end{array}
	}$
	para nos referirmos à permutação $\sigma \in S_n$.
\end{definition}

\begin{definition}
	Definimos as permutações $\phi, \psi \in S_3$, onde
	\begin{align*}
		\phi &\eqass \paren{\begin{array}{ccc} 1 & 2 & 3 \\ 2 & 1 & 3 \end{array}} \\
		\psi &\eqass \paren{\begin{array}{ccc} 1 & 2 & 3 \\ 2 & 3 & 1 \end{array}}
	\end{align*}
\end{definition}

\begin{exercise}
	\topic{grupos!permutações}
	Calcule $\psi \compose \psi^2$ e justifique que é igual à $\psi^2 \compose \psi$.
\end{exercise}

\begin{exercise}
	\topic{grupos!permutações}
	Calcule $\phi \compose \psi^2$ e a $\psi^2 \compose \phi$.
\end{exercise}

\begin{exercise}
	\topic{grupos!permutações}
	\topic{grupos!associatividade}
	\topic{grupos!comutatividade}
	\topic{grupos!identidade}
	\topic{grupos!inversos}
	Prove ou refute a validade das seguintes propriedades para a operação $\compose$ sobre $S_3$
	\begin{itemize}
		\item Associatividade
		\item Comutatividade
		\item Tem identidade
		\item Tem inversa
	\end{itemize}
\end{exercise}

\begin{exercise}
	\topic{grupos!permutações}
	\topic{grupos!inversos}
	Calcule as inversas de cada um dos elementos de $S_3$.
\end{exercise}