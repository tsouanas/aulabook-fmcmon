\begin{exercise}
\topic{indução} 
	Exponenciação nos números naturais é definida pela seguinte recursão no $m$:
	$$
		\begin{aligned}
			n^0 &= 1, && (e1)\\
			n^{Sm} &= n^m \cdot n. && (e2)
		\end{aligned}
	$$
	Mostre que ela satisfaz as seguintes identidades (para $n \neq 0$):
	$$
		\begin{aligned}
			n^{(m + k)} &= m^m \cdot n^k,\\
			n^{(m \cdot k)} &= (n^{m})^k.
		\end{aligned}
	$$
\end{exercise}

\begin{exercise}
\topic{recursão}
	Defina recursivamente o fatorial
	$$
		f(n) = 1 \cdot 2 \cdots (n -1) \cdot n.
	$$
\end{exercise}

\begin{exercise}
\topic{naturais!Peano}
	Suponha que $(\nats_1, 0_1, S_1)$ e $(\nats_2, 0_2, S_2)$ são dois sistemas Peano, $+_1, \cdot_1, +_2, \cdot_2$ são as funções de adição e multiplicação nesses sistemas, e $\pi : \nats_1 \bijto \nats_2$ é o isomorfismo ``canônico'' entre eles definido por
	$$
		\begin{aligned}
			\pi(0_1) &= 0_2,\\
			\pi(S_1n) &= S_2\pi(n) \ (n \in \nats_1).
		\end{aligned}
	$$
	Mostre que $\pi$ é um isomorfismo com respeito à adição e também à multiplicação, ou seja, para todos $n, m \in \nats_1$,
	$$
		\pi(n +_1 m) = \pi(n) +_2 \pi(m),\ \pi(n \cdot_1 m) = \pi(n) \cdot_2 \pi(m).
	$$
\end{exercise}

\begin{definition}
	A relação de ordem $\leq$ nos números naturais é definida pela equivalência
	$$
		n \leq m \iffdf (\exists s)[n + s = m].
	$$
\end{definition}

\begin{exercise}
	Suponha que $(\nats_1, 0_1, S_1)$ e $(\nats_2, 0_2, S_2)$ são dois sistemas Peano, $\leq_1, \leq_2$ são as respectivas boas ordens e $\pi : \nats_1 \bijto \nats_2$ é o isomorfismo canônico. Mostre que $\pi$ é ordem-preservante, ou seja, para todos $n, m \in \nats_1$,
	$$
		n \leq_1 m \iff \pi(n) \leq_2 \pi(m).
	$$
\end{exercise}
