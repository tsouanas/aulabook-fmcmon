\begin{definition}
	Seja $\tup{G, \, * \,, \, e \,, \;\inv{}}$ um grupo, e $H$ é um subconjunto de $G$ que possui as seguintes propriedades:
	\begin{itemize}
		\item $H$ é fechado sobre $*$
		\item $H$ é fechado sobre $\,\inv{}$
	\end{itemize}
	Dizemos que $H$ é um \emph{subgrupo} de $G$.
\end{definition}

\begin{exercise}
	\topic{grupos!subgrupo}
	Para cada item, mostre que $H$ é um subgrupo de $G$.
	\begin{itemize}
		\item $G = \tup{\reals, +}$, $H = \setst{\textrm{log}\ a}{a \in \rats, a > 0}$
		\item $G = \tup{\nonzero{\reals}, \, \ntimes \, }$, $H = \setst{2^n \stimes 3^m}{n, m \in \ints}$ 
	\end{itemize}
\end{exercise}

\begin{exercise}
	\topic{grupos!subgrupo}
	\label{centersub}
	Denotamos por \emph{centro} do grupo $\tup{G, *}$ o conjunto dos elementos de $G$ que comutam com todos os elementos de $G$. Ou seja,
		\[ C = \setst{a \in G}{\textrm{para todo}\ x \in G,\ a \stimes x = x \stimes a} \]
	Mostre que $C$ é subgrupo de $G$.
\end{exercise}

\begin{exercise}
	\topic{grupos!subgrupo}
	Seja $\tup{G, *}$ um grupo, e $H$, $K$ subconjuntos de $G$. Mostre que, se $H$ e $K$ são subgrupos de $G$, então $H \cap K$ também é subgrupo de $G$.
\end{exercise}

\begin{exercise}
	\topic{grupos!tabela de Cayley}
	\label{dihedral4}
	Calcule a tabela de Cayley para o grupo $G = \fsset{e, a, b, b^2, b^3, a \stimes b, a \stimes b^2, a \stimes b^3}$ onde
		\[ a^2 = e \;\;\; b^4 = e \;\;\; b \stimes a = a \stimes b^3 \]
\end{exercise}

\begin{exercise}
	\topic{grupos!diagrama de Cayley}
	Monte o diagrama de Cayley para o grupo $G$ do exercício \ref{dihedral4}.
\end{exercise}