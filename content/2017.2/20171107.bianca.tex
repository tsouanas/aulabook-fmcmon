Sejam $G, H$ e $K$ grupos. Prove os seguintes:

\begin{exercise}
\topic{grupos!homomorfismo}
	Se $\varphi: G \to H$ e $\psi: H \to K$ são homomorfismos, então sua composição $\psi \fcompose \varphi: G \to K$ é um homomorfismo.
\end{exercise}

\begin{exercise}
\topic{grupos!homomorfismo}
	Para qualquer grupo $G$, a função $\varphi: G \to G$ dada por $\varphi(x) = e$ é um homomorfismo.
\end{exercise}

\begin{exercise}
\topic{grupos!homomorfismo}
	A função $\varphi: G \to G$ dada por $\varphi(x) = x^2$ é um homomorfismo sse $G$ é abeliano.
\end{exercise}

\begin{exercise}
\topic{grupos!normais}
	Seja $H$ um subgrupo de $G$. $H$ é normal sse ele tem a seguinte propriedade:
	Para todos os $a, b \in G$, $ab \in H$ sse $ba \in H$.
\end{exercise}

\begin{exercise}
\topic{grupos!normais}
	Se $a$ é um elemento arbitrário de $G$, \textless $a$\textgreater \ é um subgrupo normal de $G$ sse $a$ tem a seguinte propriedade:
	Para todo $x \in G$, existe um inteiro positivo $k$ tal que $xa = a^k x$.
\end{exercise}

\begin{exercise}
\topic{anéis}
	Prove que em qualquer anel, $a(b - c) = ab - ac$ e $(b - c)a = ba -ca$.
\end{exercise}

\begin{exercise}
\topic{anéis}
	Prove que em qualquer anel, se $ab = -ba$, então $(a + b)^2 = (a - b)^2 = a^2 + b^2$.
\end{exercise}
