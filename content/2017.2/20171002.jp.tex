\begin{exercise}
	\topic{indução}
	\topic{grupos!potências}
	Recorde a definição para $a^n$:
	\begin{align*}
		a^0 &= e \\
		a^{n+1} &= a \cdot a^n
	\end{align*}

	Abaixo temos uma definição alternativa, executando o passo recursivo à esquerda:
	\begin{align*}
		a^0 &= e \\
		a^{n+1} &= a^n \cdot a
	\end{align*} 

	Mostre que essa nova definição é equivalente à usual. 
\end{exercise}

\begin{exercise}
	\topic{grupos}
	Prove que, se $a$ é o único elemento de $G$ com ordem $k$, então $a$ está no centro de $G$.
	(Recorde a definição de centro dada no exercício \ref{centersub})
\end{exercise}