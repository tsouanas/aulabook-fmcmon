\begin{exercise}
\topic{congruência!teorema!Chinês do resto}
    Usando o Teorema Chinês do Resto, ache todos os inteiros $x\in\ints$ que satisfazem o sistema de congruências:
    $$
    \left\{
        \begin{aligned}
            x&\cong 2\pmod 9\\
            x&\cong 1\pmod 5\\
            x&\cong 2\pmod 4
        \end{aligned}
    \right.
    $$
\end{exercise}

\begin{exercise}
\topic{congruência!teorema!Chinês do resto}
    Ache todos os inteiros $x\in\ints$ com $\abs x < 64$ que satisfazem o sistema de congruências:
    $$
    \left\{
        \begin{aligned}
            x &\cong 1\pmod 3\\
            3x&\cong 1\pmod 4\\
            4x&\cong 2\pmod 5
        \end{aligned}
    \right.
    $$
\end{exercise}

\begin{exercise}
\topic{teoria de conjuntos}
	Sejam $a,b$ conjuntos. Mostre que $\set {b, \set {\emptyset, {\set a}}}$ é conjunto.
\end{exercise}

\begin{exercise}
\topic{teoria de conjuntos}
	Sejam $a,b,c,d$ conjuntos. Mostre pelos axiomas que os seguintes também são:
	$$
	\begin{aligned}
		A &= \set {a, b, c, d}\\
		B &= \set {a, b, \set {c, d}}\\
		C &= \setlst {x} {x\subseteq a\union b\union c\union d \ \&\ x \text{ tem exatamente } 2 \text { membros}}.
	\end{aligned}
	$$
\end{exercise}
