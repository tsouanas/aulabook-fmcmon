\begin{exercise}
\topic{grupos!comutatividade}
\topic{grupos!associatividade}
	Se uma nova adição para números reais, denotada pelo símbolo temporário $\boxplus$, é definida por
	$$
		\alpha \boxplus \beta = 2\alpha + 2\beta,
	$$
	ela é comutativa? É associativa?
\end{exercise}

\begin{exercise}
\topic{grupos!comutatividade}
\topic{grupos!associatividade}
	Se uma nova adição para números reais, denotada pelo símbolo temporário $\boxplus$, é definida por
	$$
		\alpha \boxplus \beta = 2\alpha + \beta,
	$$
	ela é comutativa? É associativa?
\end{exercise}

\begin{exercise}
\topic{grupos!comutatividade}
\topic{grupos!associatividade}
	Se uma operação para inteiros positivos, denotada pelo símbolo temporário $*$, é definida por
	$$
		\alpha * \beta = \alpha^{\beta},
	$$
	ela é comutativa? É associativa?
\end{exercise}

\begin{exercise}
\topic{grupos!comutatividade}
\topic{grupos!associatividade}
	Se uma operação para pares ordenados de números reais, denotada pelo símbolo temporário $\boxdot$, é definida por
	$$
		(\alpha,\beta) \boxdot (\gamma, \delta) = (\alpha \gamma - \beta \delta, \alpha \delta + \beta \gamma),
	$$
	ela é comutativa? É associativa?
\end{exercise}

\begin{exercise}
\topic{grupos!comutatividade}
\topic{grupos!associatividade}
	Se uma operação para pares ordenados de números reais, denotada mais uma vez pelo $\boxdot$, é definida por
	$$
		(\alpha,\beta) \boxdot (\gamma,\delta) = (\alpha \gamma, \alpha \delta + \beta),
	$$
	ela é comutativa? É associativa?
\end{exercise}
