\begin{definition}
    Definimos os \emph{numerais} \texttt{Nat} para representar os números 
    naturais com uma definição indutiva:

    \begin{itemize}
        \item 0 é um \texttt{Nat};
        \item Se $n$ é um \texttt{Nat}, então $Sn$ é um \texttt{Nat};
        \item Nada mais é \texttt{Nat}.
    \end{itemize}

    Alternativamente, podemos usar a notação BNF para definir o \texttt{Nat} como
    segue:
    $$
    \tup{\texttt{Nat}} \mathrel{::=} 0\ |\ S\tup{\texttt{Nat}}
    $$
\end{definition}

\begin{exercise}
    Defina recursivamente as operações de adição e multiplicação no \texttt{Nat}.
\end{exercise}

\begin{exercise}
    Descreva formalmente (com fórmulas de lógica), o que significa afirmar que a
    adição (definida no exercício anterior) é associativa e em seguida demonstre
    que ela realmente é.
\end{exercise}
