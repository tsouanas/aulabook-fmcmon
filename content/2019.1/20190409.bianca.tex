\begin{exercise}
\topic{relações!equivalência}
	Sejam $\sim$ uma relação de equivalência num conjunto $X$, e $x, y \in X$. O.s.s.e.:
	\begin{enumerate}[(i)]
		\item $x \sim y$
		\item $\eqclassimp x = \eqclassimp y$
		\item $\eqclassimp x \intersection \eqclassimp y \neq \emptyset$
	\end{enumerate}
\end{exercise}

\begin{exercise}
\topic{relações}
    Seja $f : A \to B$ uma função. Defina $\sim$ por $a \sim b \text{ sse } f(a) = f(b)$. Prove que $\sim$ é uma relação de equivalência em $A$ e descreva suas classes de equivalência.
\end{exercise}

\begin{exercise}
\topic{relações}
    No conjunto $\reals$, defina $\sim$ por $a \sim b \text{ sse } a - b \in \ints$. Mostre que $\sim$ é uma relação de equivalência e descreva suas classes de equivalência.
\end{exercise}

\begin{exercise}
\topic{relações}
	Seja $R$ uma relação binária num conjunto $A$. O.s.s.e.:
	\begin{enumerate}[(i)]
		\item $R$ é uma relação de equivalência;
		\item $R$ é reflexiva e circular;
		\item $R$ é reflexiva e right-euclidean.
	\end{enumerate}
\end{exercise}
