\begin{definition}
    Sejam $a, b \in \ints$ e $m \in \nats_{>0}$. Dizemos que $a$ é
    \emph{congruente a} $b$ \emph{módulo} $m$ se, e somente se, $m \divs (a-b)$.
    Denotamos essa relação por $a \congmod m b$ ou $a \cong_m b$.
\end{definition}

\begin{exercise}
    Se $a, b$ são inteiros, temos que $a \cong_m b$ se, e somente se, $a = b +
    mk$, para algum $k \in \ints$.
\end{exercise}

\begin{exercise}
    Sejam $a,b,c,m \in \ints$, com $m > 0$. As seguintes sentenças são
    verdadeiras:
    \begin{enumerate}[(i)]
        \item $a \cong_m a$;
        \item Se $a \cong_m b$, então $b \cong_m a$;
        \item Se $a \cong_m b$ e $b \cong_m c$, então $a \cong_m c$.
    \end{enumerate}
\end{exercise}

\begin{homework}
    Sejam $a,b,c \in \ints$ e $m \in \nats_{>0}$ tal que $a \cong_m b$, então:
    \begin{enumerate}[(i)]
        \item $a+c \cong_m b+c$
        \item $a-c \cong_m b-c$
        \item $ac \cong_m bc$
    \end{enumerate}
\end{homework}
