%%% Local Variables:
%%% mode: latex
%%% TeX-master: "../../aulabook"
%%% End:
\begin{definition}
    Se $a$ e $b$ são dois inteiros com $a \cong_m b$, dizemos que $b$ é resíduo
    de $a$ módulo $m$.
\end{definition}

\begin{definition}
    O conjunto dos inteiros $\set{r_0, \dots, r_{s-1}}$ é um \emph{sistema
    completo de resíduos módulo m} se
    \begin{enumerate}[(i)]
    \item $r_i \not\cong_m r_j$ para $i \neq j$
    \item para todo inteiro $n$ existe um $r_i$ tal que $n \cong_m r_i$.
    \end{enumerate}
\end{definition}

\begin{exercise}
    Se $k$ inteiros $r_0, \dots, r_{k-1}$ formam um sistema completo de resíduos
    módulo $m$, então $k = m$.
\end{exercise}
