\begin{exercise}

\topic{grupos}

Seja $\tup{G; e, *}$ um conjunto estruturado satisfazendo os seguintes axiomas:

\begin{itemize}

\item[(G0)] $(\forall a, b \in G)[a * b \in G]$
\item[(G1)] $(\forall a, b, c \in G)[a * (b * c) = (a * b) * c]$
\item[(G2L)] $(\forall a \in G)[e * a = a]$
\item[(G3L)] $(\forall a \in G)(\exists y \in G)[y * a = e]$

\end{itemize}

Prove que $\tup{G; e, *}$ é um grupo.

\end{exercise}


\begin{exercise}

\topic{grupos}

Se trocarmos o (G2L) acima pelo
	
\begin{itemize}
\item[(G2R)] $(\forall a \in G)[a * e = a]$
\end{itemize}

Ainda podemos afirmar que $\tup{G; e, *}$ é um grupo?

\end{exercise}


\begin{exercise}

\topic{grupos!subgrupos}%
\topic{grupos!geradores}%

Seja $G$ um grupo e $a \in G$. Prove que $\tup{a}$ é um subgrupo de $G$.

\end{exercise}


\begin{exercise}

\topic{monoides!homomorfismo}%

Sejam $M, N$ monoides, e $\phi : M \surjto N$ sobrejetiva que respeita a operação. Prove que $\phi$ é homomorfismo de monoides. 

\end{exercise}


