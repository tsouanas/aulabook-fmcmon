\begin{exercise}
    \topic{teoria de números!congruência}
    Mostre que, se $n >0$ é tal que $(n-1)! \congmod n {-1}$, então $n$ é primo.
\end{exercise}

\begin{exercise}
\topic{teoria de números!congruência}
    Sejam $p$ primo e $a>0$. Mostre que $a^p \congmod p a$.
\end{exercise}

\begin{exercise}
    \topic{ordens parciais}
    Sejam $P$ poset, $A \subseteq P$ e $u \in A^{\mathrm{U}}\cap A$. Mostre que $u = \mathrm{lub}A = \mathrm{max}A$.  
\end{exercise}

\begin{exercise}
    \topic{ordens parciais}
    Seja $A$ um conjunto e $R$ uma relação binária em $A$ irreflexiva e transitiva. Mostre que a relação

    $$x \le y \reldef x\le y \text{ ou } x=y$$

    é uma ordem parcial.
\end{exercise}