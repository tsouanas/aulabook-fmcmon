\begin{exercise}
\topic{conjuntos!teoria axiomática}
    Considere o axioma seguinte:
    %
    \begin{description}
        \item[CONS] $\forall h \forall t \exists s \forall x (x \in s \leftrightarrow x=h \lor x \in t)$
    \end{description}
    %
    Mostre que, no sistema ZF1+ZF2$^*$+ZF3+ZF4+ZF5+ZF6, o ZF2 é um teorema.
\end{exercise}

\begin{exercise}
\topic{conjuntos!teoria axiomática}
    Seja $a$ conjunto. Prove que, no sistema ZF1+ZF2+ZF4+ZF5+ZF6, $\set{a}$ também o é.
\end{exercise}

\begin{exercise}
\topic{conjuntos!teoria axiomática}
    Sejam $a$, $b$, $c$ e $d$ conjuntos. Prove, pelos axiomas ZFC, que os seguintes também o são:
    %
    \begin{enumerate}[a)]
        \item $A = \set{a,b,c,d}$;
        \item $B = \set{a,b,\set{c,d}}$;
        \item $C = \set{x \st x \subseteq a \cup b \cup c \cup d \mathrel{\&} \text{$x$ tem exatamente dois membros}}$.
    \end{enumerate}
\end{exercise}