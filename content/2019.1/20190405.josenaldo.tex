\begin{definition}
    Dados $A$ e $B$ conjuntos, e $R$ e $S$ relações de $A$ para $B$, definem-se:
    
        $$R \le S \reldef (\forall a \in A )(\forall b \in B )[a \mathrel R b \implies  a \mathrel S b]$$
        $$\text{Dom}(R) \eqdef \set{a \in A \mid \text{existe $b \in B$ tal que $a \mathrel R b$}}$$
        $$\text{Ran}(R) \eqdef \set{b \in B \mid \text{existe $a \in A$ tal que $a \mathrel R b$}}$$
    
\end{definition}

\begin{exercise}
\topic{relações}
    \label{exe:04052019_1}
    Dados $A$ e $B$ conjuntos, e $R$ uma relação de $A$ para $B$, prove que $\text{Dom}(R) \subseteq \text{Ran}(R^{\partial})$.
\end{exercise}

\begin{homework}
    Prove a inclusão contrária no Exercício \ref{exe:04052019_1}.
\end{homework}

\begin{exercise}
\topic{relações}
    \label{exe:04052019_2}
    Sejam $A$ e $B$ conjuntos, e $R$ e $S$, relações de $A$ para $B$. Mostre que: $$R \le S \iff R^{\partial} \le S^{\partial}$$
\end{exercise}

\begin{homework}
\topic{relações}
    No contexto do Exercício \ref{exe:04052019_2}, prove que $R^{\partial} \le S^{\partial} \iff R \le S$.
\end{homework}

\begin{definition}
    Dados $A$ e $B$ conjuntos, define-se a relação $0_{A, B}$ de $A$ para $B$ tal que, para todos $a \in A$ e $b \in B$,
    $$0_{A, B}(a,b) \reldef a \ne a$$
\end{definition}

\begin{exercise}
\topic{relações!composição}
\label{exe:04052019_3}
    Sejam $A$, $B$ e $C$ conjuntos, e $R$ e $S$ relações de $A$ para $B$ e $B$ para $C$, respectivamente. Prove que:
    $$\text{Ran}(R) \cap \text{Dom}(S) = \emptyset \implies R \rcom S = 0_{A,C}$$
\end{exercise}

\begin{homework}
\topic{relações!composição}
    Prove a afirmação conversa do Exercício \ref{exe:04052019_3}.
\end{homework}