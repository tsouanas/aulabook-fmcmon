\begin{definition}
    Os n\'umeros de {\em Fibonacci} e os n\'umeros de {\em Lucas} s\~ao
    definidos recursivamente como segue, respectivamente
    \begin{align*}
        F_0     &= 0             & L_0     &= 2 \\
        F_1     &= 1             & L_1     &= 1 \\
        F_{n+2} &= F_{n+1} + F_n & L_{n+2} &= L_{n+1} + L_n \\
    \end{align*}
\end{definition}

\begin{exercise}
    Definimos a fun{\c c}\~ao $l \colon \nats \setminus \{0\} \to \nats$ pela
    equa{\c c}\~ao
    $$
    l(n)=F_{n-1}+F_{n+1}
    $$
    Demonstre, para todo $n \in \nats_{\geq 1}$, que $l(n)=L_n$.
\end{exercise}

\begin{exercise}
    Sejam $a,b \in \ints$ e $k,m \in \nats_{>0}$, tal que $a \cong_m b$.
    Demonstre que $a^k \cong_m b^k$.
\end{exercise}
