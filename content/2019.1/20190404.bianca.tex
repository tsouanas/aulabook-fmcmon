\begin{exercise}
\topic{relações!composição}
Seja $R$ uma preordem num conjunto $A$. Prove que $R$ é idempotente, ou seja, $R = R \rcompose R$.
\end{exercise}

\begin{exercise}
\topic{relações}
	Proposição: Seja $X \neq \emptyset$ e $\sim$ uma relação no $X$. Se $\sim$ é simétrica e transitiva, então ela é reflexiva.

	Prova: Como ela é simétrica, de $x \sim y$, concluímos que $y \sim x$ também. Usando a transitivdade, de $x \sim y$ e $y \sim x$, concluímos a $x \sim y$, que mostra que $\sim$ é reflexiva também.

	Ache o erro na prova acima e prove que a proposição é falsa.
\end{exercise}

\begin{exercise}
\topic{relações!composição}
Considere a $\rcompose$ como uma operação binária nas relações binárias num conjunto $A$. Ela tem identidade? Ou seja, existe alguma relação binária $I$ no $A$ tal que para toda relação $R$ no $A$,
$$
    I \rcompose R = R = R \rcompose I \text{?}
$$

Se sim, defina essa relação $I$ e prove que realmente é. Se não, prove que não existe.
\end{exercise}

\begin{exercise}
\topic{relações!composição}
Defina formalmente as ``potências'' $R^n$ de uma dada relação binária $R$ em um conjunto, informalmente definida por:

$$
    x(R^n)y \iffpseudodf x (R \rcompose \cdots \rcompose R) y, 
$$ 

válida para todo $n \in \nats$.
\end{exercise}

\begin{exercise}
\topic{relações!composição}
Seja $S$ uma relação binária no $\reals$ tal que

$$
    (S \rcompose S^{\partial}) \text{ é irreflexiva}.
$$

Qual é o gráfico da $S$? Prove tua resposta.
\end{exercise}
