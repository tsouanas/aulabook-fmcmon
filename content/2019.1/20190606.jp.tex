\begin{exercise}
	\topic{set theory}%
	\topic{peano system}%

	Prove os seguintes itens: 
	\begin{itemize}
		\item $0 \in \mathbf{N}$
		\item $S : \mathbf{N} \to \mathbf{N}$
	\end{itemize}
\end{exercise}

\begin{homework}
	\topic{set theory}%
	\topic{peano system}%

	Prove os seguintes itens:
	\begin{itemize}
		\item $S$ é injetiva.
		\item Para todo $X \subseteq \mathbf{N}$, se $X$ satisfaz
			\begin{enumerate}[(i)]
				\item $0 \in X$
				\item se $n \in X$, então $n^+ \in X$
			\end{enumerate}
			então $X = \mathbf{N}$.
	\end{itemize}
\end{homework}

\begin{theorem} Para todos $a, b, n, m \in \nats$, se $a \cong_m b$, então $a^n \cong_m b^n$.\end{theorem}

\begin{exercise}
	\topic{number theory!modular arithmetic}

	Usando o teorema acima, prove que $13 \divides 2^{70} + 3^{70}$. 
\end{exercise}
