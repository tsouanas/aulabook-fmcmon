\begin{exercise}
\topic{funções!composição}
Prove que a composição de injeções é uma injeção; a composição de sobrejeções é uma sobrejeção; e consequentemente, a composição de bijeções é uma bijeção.
\end{exercise}

\begin{exercise}
\topic{funções!composição}
Nas questões seguintes, sejam $A,B,C$ conjuntos e sejam $f:A \to B$ e $g:B \to C$ funções.
\begin{enumerate}[(i)]
    \item Mostre que se $g \fcompose f$ é injetora, então $f$ é injetora.
    \item Mostre que se $g \fcompose f$ é sobrejetora, então $f$ é sobrejetora.
    \item Questões (i) e (ii), juntas, nos dizem que se $g \fcompose f$ é bijetora, então $f$ é injetora e $g$ é sobrejetora. A recíproca dessa afirmação é verdadeira?
\end{enumerate}
\end{exercise}

\begin{exercise}
\topic{funções}
    Sejam $A \not = \emptyset$ conjunto, $n \in \nats_{>0}$, e $I = \set{i \in \nats \st i < n}$. Considere a função $\pi : I \times A^n \to A$ definida por:
    $$
        \pi (i, \alpha) = \pi_i(\alpha) \qquad \text{(= o iésimo elemento da tupla $\alpha$)}
    $$

    \begin{enumerate}[(a)]
        \item $\pi$ é injetora?
        \item $\pi$ é sobrejetora?
    \end{enumerate}
\end{exercise}
