\begin{exercise}
\topic{relações!equivalência}
    Defina com texto completo o conjunto quociente.
\end{exercise}

\begin{exercise}
\topic{relações!equivalência}
    Defina com texto completo o que é uma partição. 
\end{exercise}

\begin{exercise}
\topic{relações!equivalência}
    Seja $\sim$ uma relação de equivalência num conjunto $A$. Prove que o conjunto quociente $\quotset{A}{\sim}$ é uma partição de $A$.
\end{exercise}

\begin{exercise}
\topic{relações!equivalência}
    Considere as relações seguintes no $(\ints \to \ints)$:
    $$
    f \sim g \iffdf (\exists u \in \ints)(\forall x \in \ints)[f(x) = g(x + u)]
    $$
    $$
    f \text{ } \rotatebox{90} {$\sim$} \rotatebox{90} {$\sim$} \text{ } g \iffdf (\exists v \in \ints)(\forall x \in \ints)[f(x) = g(x) + v]
    $$
    Prove que uma delas é uma relação de equivalência e que a outra é uma relação de ordem.
\end{exercise}
