\begin{exercise}
\topic{teoria de números!congruência}
    Sejam $a$, $b$, $k$ e $m$ inteiros tais que $m>0$ e $a \congmod m b$. 
    Mostre que $a^k \congmod m {b^k}$.
\end{exercise}

\begin{exercise}
\topic{conjuntos!teoria axiomática}
Considere o axioma seguinte:

\begin{description}
	\item[CONS] $\forall h \forall t \exists s \forall x (x \in s \leftrightarrow x=h \lor x \in t)$
\end{description}

\begin{enumerate}[a)]
\item No sistema ZF1+ZF2+CONS, prove o ZF3 como teorema.
\item Mostre que é impossível provar o CONS no sistema ZF1+ZF2+ZF3. 
\item No sistema ZF1+ZF2+ZF3+ZF4+ZF5+ZF6, prove o CONS como teorema.
\end{enumerate}
\end{exercise}

\begin{exercise}
\topic{conjuntos!cardinalidade}
Mostre que $\pset_{\infty}\nats$ não é enumerável, em que $\pset_{\infty}A = \set{X \subseteq A \st X \text{ é infinito}}$ 
para qualquer conjunto $A$. 
\end{exercise}