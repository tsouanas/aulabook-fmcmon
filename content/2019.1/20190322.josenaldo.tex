\begin{definition}
Dados $f: A \to B$ e $x \in A$, $x$ é um ponto fixo de $f$ se $f(x)=x$.
\end{definition}

\begin{definition}
Dada $f: A \to A$, definem-se as iterações $f^n$ como:
%
\begin{gather*}
f^0 = id_A\\
f^{n+1} = f \circ f^n
\end{gather*}

\end{definition}

\begin{exercise}
Sejam $f: A \to A$ e $x \in A$. Prove que:

\begin{center}
        $x$ é ponto fixo de $f$ $\iff$ para todo $n \in \nats$, $x$ é ponto fixo de $f^n$
    \end{center}
\end{exercise}

\begin{exercise}   
    Seja $f: \nats \to \nats$. Prove que:

    \begin{center}
        $f$ é sobrejetiva $\iff$ existe $g: \nats \to \nats$ tal que $f \circ g = id_A$
    \end{center}
\end{exercise}