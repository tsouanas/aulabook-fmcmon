\begin{exercise}
\topic{indução}
	Algumas definições de funções recursivas são:
	$$
		\left.
			\begin{aligned}
				x + 0 = x \quad \text{(a1)}\\
				x + Sy = S(x + y) \quad \text {(a2)}
			\end{aligned}
		\right.
		\qquad
		\left.
			\begin{aligned}
				x \cdot 0 = 0 \quad \text{(m1)}\\
				x \cdot Sy = (x \cdot y) + x \quad \text{(m2)}
			\end{aligned}
		\right.
	$$
	$$
		\left.
			\begin{aligned}
				x^0 = S0 \quad \text{(e1)}\\
				x^{Sy} = x^y \cdot x \quad \text{(e2)}
			\end{aligned}
		\right.
	$$
    Dados os lemas:
    $$
		\left.
			\begin{aligned}
			    a + b = b + a \qquad \text{(a-com)}\\
			    a \cdot (b \cdot c) = (a \cdot b) \cdot c \qquad \text{(m-ass)}
			\end{aligned}
		\right.
	$$
    prove por indução as seguintes propriedades, indicando para cada passo o que foi usado:
	$$
		\text{(i)}
			\quad a \cdot S0 = a
			\qquad
			\text{(ii)}
				\quad a^{x + y} = a^x \cdot a^y
			\qquad
			\text{(iii)}
				\quad a^{x \cdot y} = (a^x)^{y}.
	$$
\end{exercise}
