\begin{exercise}
\topic{conjunto}
	Suponha que $A, B$ e $C$ são conjuntos. Então $A \intersection (B \setminus C) = (A \intersection B) \setminus C$.
\end{exercise}

\begin{exercise}
\topic{conjunto}
	Suponha que $A, B$ e $C$ são conjuntos. Prove que se $A \subseteq C$ e $B \subseteq C$, então $A \union B \subseteq C$.
\end{exercise}

\begin{exercise}
\topic{conjunto}
	Suponha $A, B$ e $C$ conjuntos, $A \setminus B \subseteq C$ e $x$ arbitrário. Prove que se $x \in A \setminus C$, então $x \in B$.
\end{exercise}

\begin{exercise}
\topic{conjunto!produto cartesiano}
	Suponha $A, B$ e $C$ conjuntos. Prove que $A \times (B \intersection C) = (A \times B) \intersection (A \times C)$.
\end{exercise}

\begin{exercise}
\topic{indução}
	Prove que para todo inteiro positivo $n$, uma grid de $2^n$ x $2^n$ com qualquer um dos quadrados removidos pode ser coberto por peças em forma de L.
\end{exercise}

\begin{exercise}
\topic{indução}
	Use indução para mostrar que para todo $n \geq 1$
	$$
	n! \geq 2^{n - 1} 
	$$
\end{exercise}

\begin{exercise}
\topic{indução}
	Suponha as propriedades de adição para os números naturais, mas que multiplição não é conhecida. Então, o seguinte pode ser usado como uma definição recursiva de multiplicação:
	$$
		1 \cdot b = b \quad \text{(i)}
	$$
	$$
		(a + 1) \cdot b = a \cdot b + b \quad \text{(ii)}
	$$
	Prove o seguinte:
	\begin{enumerate}[(a)]
		\item $a \cdot (b + c) = a \cdot b + a \cdot c$
		\item $a \cdot 1 = a$
		\item $a \cdot b = b \cdot a$
	\end{enumerate}
\end{exercise}
