\begin{definition}[Naturais]
\topic{naturals!peano}
\topic{bnf}
Definimos o conjunto dos naturais recursivamente, da seguinte forma:
 $ \tup{\textbf{Nat}} \;::=\; O \;\; \vert \;\; S \; \tup{\textbf{Nat}} $
\end{definition}

\begin{exercise}
\topic{naturals!peano}
\topic{recursion}
Defina a operação de adição sobre os números naturais como definidos acima.
\end{exercise}

\begin{exercise}
\topic{induction}
Demonstre que a adição definida no exercício anterior é associativa.
Isso é, mostre que para todo $m, n, p$ naturais, $m + (n + p) = (m + n) + p$.
\end{exercise}

\begin{definition}[Lema 1 da adição]
	\label{lema12018}
	Para todo $n$ natural, $n + 0 = n$.
\end{definition}

\begin{definition}[Lema 2 da adição]
	\label{lema22018}
	Para todo $m, n$ naturais, $S(m + n) = m + Sn$.
\end{definition}

\begin{homework}
\topic{induction}
Demonstre \ref{lema12018} (p.~\pageref{lema12018}) e \ref{lema22018} (p.~\pageref{lema22018}).
\end{homework}

\begin{exercise}
\topic{induction}
\topic{lemmas}
Demonstre que a adição é comutativa (ou seja, que para todo $m, n$ naturais, $m + n = n + m$).
Use coisas que demonstramos anteriormente.
\end{exercise}

\begin{homework}
Defina a multiplicação, e prove sua associatividade e comutatividade
\end{homework}
