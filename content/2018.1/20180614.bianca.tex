\begin{exercise}
\topic{anéis}
Seja $\langle R;0;+;\cdot \rangle$ anel. Prove que:
\begin{enumerate}[(i)]
    \item $0x = 0 = x0$
    \item $(-x)y = -(xy) = x(-y)$
    \item $(-x)(-y) = xy$
\end{enumerate} 
\end{exercise}

\begin{exercise}
\topic{anéis}
Um anel $\langle B;0;+;\cdot \rangle$ com unidade é booleano sse $p^2=p$ para todo $p \in B$. Prove que:
\begin{enumerate}[(i)]
    \item $p + p = 0$ para todo $p \in B$. (Dica: calcule o $(p + q)^2$)
    \item $B$ é um anel comutativo.
\end{enumerate}
\end{exercise}

\begin{definition}[Domínio de integridade]
Um anel comutativo $D$ tal que para todo $x,y \in D$,
$$
    \text{se } xy = 0 \text{, então } x = 0 \text{ ou } y = 0 \quad \text(NZD)
$$
é chamado domínio de integridade.
\end{definition}

\begin{definition}[Domínio de cancelamento]
Um anel comutativo $D$ tal que para todo $a,x,y \in D$,
$$
    ax = ay \text{ \& } a \not = 0 \implies x = y \quad \text(CL)
$$
é chamado domínio de cancelamento.
\end{definition}

\begin{exercise}
\topic{anéis}
Mostre que:
$$
    D \text{ é um domínio de integridade } \iff D \text { é um domínio de cancelamento}.
$$ 
\end{exercise}
