\begin{exercise}
\topic{funções}
\label{ex1}
    Cada um dos seguintes é uma função $f : \reals \to \reals$. Determine:
    \begin{enumerate}[(a)]
        \item se $f$ é ou não injetiva, e
        \item se $f$ é ou não sobrejetiva.
    \end{enumerate}
    Prove sua resposta em qualquer um dos casos.

    \begin{enumerate}[(i)]
        \item $f(x) = 2x$
        \item $f(x) = x^2$
        \item $f(x) = 3x + 4$
        \item $f(x) = \left \{ \begin{aligned}
                        &2x &\text{se $x$ é um inteiro}\\
                        &x  &\text{caso contrário}
                         \end{aligned}
                      \right.$
    \end{enumerate}
\end{exercise}

\begin{exercise}
\topic{funções}
    $A$ e $B$ são conjuntos e $A \times B$ denota o conjunto de todos os pares ordenados $(x,y)$, onde $x \in A$ e $y \in B$. Determine se cada um das funções seguintes é ou não (a) injetiva e (b) sobrejetiva. Proceda como no exercício \ref{ex1}.
    \begin{enumerate}[(i)]
        \item $f : A \times B \to A$, definida por $f(x,y) = x$.
        \item $f : A \times B \to B \times A$, definida por $f(x,y) = (y,x)$
        \item $f : A \to A \times B$, definida por $f(x) = (x,b)$, onde $b$ é um elemento fixo de $B$.
    \end{enumerate}
\end{exercise}

\begin{exercise}
\topic{funções}
    Sejam $A,B,C$ conjuntos, $f : A \to B$ e $g : B \to C$.
    \begin{enumerate}[(i)]
        \item Prove que se $g \fcompose f$ é injetiva, então $f$ é injetiva.
        \item Prove que se $g \fcompose f$ é sobrejetiva, então $f$ é sobrejetiva.
    \end{enumerate}
\end{exercise}
