\begin{exercise}
\topic{relações}
	Proposição: Seja $X \neq \emptyset$ e $\sim$ uma relação no $X$. Se $\sim$ é simétrica e transitiva, então ela é reflexiva.

	Prova: Como ela é simétrica, de $x \sim y$, concluímos que $y \sim x$ também. Usando a transitivdade, de $x \sim y$ e $y \sim x$, concluímos a $x \sim y$, que mostra que $\sim$ é reflexiva também.

	Ache o erro na prova acima e prove que a proposição é falsa.
\end{exercise}

\begin{exercise}
\topic{relações}
	Seja $R$ uma relação binária num conjunto $A$. O.s.s.e.:
	\begin{enumerate}[(i)]
		\item $R$ é uma relação de equivalência;
		\item $R$ é reflexiva e circular;
		\item $R$ é reflexiva e left-euclidean.
		\item $R$ é reflexiva e right-euclidean.
	\end{enumerate}
\end{exercise}

\begin{exercise}
\topic{relações!equivalência}
	Sejam $\sim$ uma relação de equivalência num conjunto $X$, e $x, y \in X$. O.s.s.e.:
	\begin{enumerate}[(i)]
		\item $x \sim y$
		\item $\eqclassimp x = \eqclassimp y$
		\item $\eqclassimp x \intersection \eqclassimp y \neq \emptyset$
	\end{enumerate}
\end{exercise}
