\begin{exercise}
\topic{funções!composição}
    Sejam $A,B$ conjuntos diferentes, e $f : A \to B$. Para cada uma das igualdades, decida se ela é válida ou não, justificando sua resposta.
    \begin{enumerate}[(1)]
        \item $f = f \fcompose$ id$_A$
        \item $f = f \fcompose$ id$_B$
        \item $f =\ $id$_A \fcompose f$
        \item $f =\ $id$_B \fcompose f$
    \end{enumerate}
\end{exercise}

\begin{exercise}
\topic{funções!composição}
    $A,B,C$ são três conjuntos diferentes; $f,g,h$ e $k$ são funções com os seguintes domínios e contra-domínios:
    $$
    \left.
        f : A \to B,
    \right.
    \qquad
    \left.
        g : B \to A,
    \right.
    \qquad
    \left.
        h : A \to C,
    \right.
    \qquad
    \left.
        k : C \to B.
    \right.
    $$
    Duas das expressões abaixo fazem sentido. Encontre-as e determine seus respectivos domínios e contra-domínios.
    \begin{enumerate}[(a)]
        \item $k \fcompose h \fcompose g \fcompose f$
        \item $k \fcompose f \fcompose g$
        \item $g \fcompose f \fcompose g \fcompose k \fcompose h$
    \end{enumerate}
\end{exercise}

\begin{exercise}
\topic{funções!composição}
    $A$ e $B$ são conjuntos, $f : A \times B \to B \times A$ é definida pela $f(x,y) = (y,x)$.

    Encontre a $g \fcompose f$ e seu domínio e contra-domínio.
\end{exercise}

\begin{exercise}
\topic{funções!composição}
\hfill

$ f : \reals \to \reals$ é definida pela $f(x) = sen(x)$.

$ g : \reals \to \reals$ é definida pela $g(x) =$ e$^x$.

\hfill

Encontre $f \fcompose g$ e $g \compose f$.
\end{exercise}

\begin{exercise}
\topic{funções!composição}
    $A = \set {a,b,c,d}$; $f$ e $g$ são funções de $A$ para $A$; elas são definidas, na forma tabular, pelas
    $$
    \begin{aligned}
        f &= \paren{\begin{array}{cccc} a & b & c & d \\ a & c & a & c \end{array}}
       &g &= \paren{\begin{array}{cccc} a & b & c & d \\ b & a & b & a \end{array}}
    \end{aligned}
    $$
    Encontre $f \fcompose g$ e $g \fcompose f$ na mesma forma tabular.
\end{exercise}

\begin{exercise}
\topic{funções!composição}
    Verdade ou falso? (Prove sua resposta).

    Se $g \fcompose f$ é constante, então pelo menos uma das $f$, $g$ também é.
\end{exercise}
