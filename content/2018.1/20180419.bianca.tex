Nas questões \ref{finj} a \ref{gofbi}, sejam $A,B,C$ conjuntos e sejam $f : A \to B$ e $g : B \to C$ funções.

\begin{exercise}
\topic{funções}
\label{finj}
    Prove que se $g \fcompose f$ é injetiva, então $f$ é injetiva.
\end{exercise}

\begin{exercise}
\topic{funções}
\label{gsob}
    Prove que se $g \fcompose f$ é sobrejetiva, então $f$ é sobrejetiva.
\end{exercise}

\begin{exercise}
\topic{funções}
\label{gofbi}
    As questões \ref{finj} e \ref{gsob}, juntas, nos dizem que se $g \fcompose f$ é bijetiva, então $f$ é injetiva e $g$ é sobrejetiva. A recíproca dessa proposição é verdade? Se $f$ é injetiva e $g$ sobrejetiva, a $g \fcompose f$ é bijetiva?
\end{exercise}

\begin{exercise}
\topic{funções}
    Sejam $f : A \to B$ e $g : B \to A$ funções. Suponha que $y = f(x)$ sse $x = g(y)$. Prove que $f$ é bijetiva, e $g = f^{-1}$.
\end{exercise}

\begin{exercise}
\topic{funções}
    Para toda $f : X \to Y$, e todos $A, B \subseteq Y$,
    \begin{enumerate}[(a)]
        \item $f^{-1}[A \union B] = f^{-1}[A] \union f^{-1}[B]$
        \item $f^{-1}[A \intersection B] = f^{-1}[A] \intersection f^{-1}[B]$
        \item $f^{-1}[A \setminus B] = f^{-1} [A] \setminus f^{-1}[B]$
    \end{enumerate}
\end{exercise}

\begin{exercise}
\topic{funções}
    Para toda $f : X \to Y$ e todas as sequências de conjuntos $A_n \subseteq X$, $B_n \subseteq Y$,
    \begin{enumerate}[(a)]
        \item $f^{-1}[\Union^{\infty}_{n=0} B_n] = \Union^{\infty}_{n=0} f^{-1}[B_n]$
        \item $f^{-1}[\Intersection^{\infty}_{n=0} B_n] = \Intersection^{\infty}_{n=0} f^{-1}[B_n]$
        \item $f[\Union^{\infty}_{n=0} A_n] = \Union^{\infty}_{n=0} f[A_n]$
    \end{enumerate}
\end{exercise}
