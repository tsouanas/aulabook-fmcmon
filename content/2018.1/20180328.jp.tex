\begin{definition}[Partição]
Dado um conjunto $A$, uma \emph{partição} de $A$ é um conjunto $\mathcal{A}$ de ubconjuntos de $A$ satisfazendo as seguintes propriedades:
\begin{enumerate}[i]
  \item $\Union \mathcal{A} = A$
  \item para todos $X,Y \in \mathcal{A}$, $X \inter Y = \varnothing$
  \item $\varnothing \not\in \mathcal{A}$
\end{enumerate}
\end{definition}

\begin{exercise}
\topic{conjuntos}
Dado $A = \set{0,1,2,3,4,5,6}$, determine para cada item abaixo se o mesmo configura uma partição de $A$.

\begin{itemize}
  \item $\mathcal{A}_1 = \set{\set{0,1,2},\set{3},\set{4,5,6}}$
  \item $\mathcal{A}_2 = \set{\set{0,1,2},\set{2,3,4},\set{4,5,6}}$
  \item $\mathcal{A}_3 = \set{\set{0,1,2},\set{3,4},\set{5,6,7}}$
  \item $\mathcal{A}_4 = \set{\set{0,1},\varnothing,\set{2,3,4},\set{5,6}}$
\end{itemize}
\end{exercise}

\begin{exercise}
\topic{conjuntos}
\topic{relações}
Dado $\mathcal{A} = \set{\set{0,1},\set{2},\set{3,4,5}}$ partição de $A = \set{01,2,3,4,5}$, determine a relação de equivalência sobre $A$ induzida por $\mathcacal{A}$
\end{exercise}


\begin{exercise}
\topic{relações}
Verifique a seguinte asserção: dadas duas relações de equivalência $R$ e $S$ sobre um conjunto $A$, sua composição é também uma relação de equivalência.
\end{exercise}
