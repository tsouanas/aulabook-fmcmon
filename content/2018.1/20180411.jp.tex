\begin{definition}[Função constante]
Uma função $f : A \to B$ é dita constante se, para todos $x,y \in A$, temos $f(x) = f(y)$.
\end{definition}

\begin{exercise}
\topic{funções}
Tome a seguinte definição alternativa para função constante:
Uma função $f : A \to B$ é dita constante se existe um $y \in B$ tal que para todo $x \in A$, temos $f(x) = y$.

Mostre que as duas definições são equivalentes.
\end{exercise}

\begin{exercise}
\topic{cálculo lambda}}
Para cada item abaixo, construa um $\lambda$-termo com o tipo correspondente.
\begin{itemize}
    \item \reals \to \reals
    \item \reals \to (\reals \to \reals)
    \item (\reals \to \reals) \to \reals
    \item (\reals \to \reals) \to (\reals \to \reals)
    \item (\reals^2 \to \reals) \to \reals
    \item (\reals^2 \to \reals) \to (\reals \to (\reals \to \reals))
\end{itemize}
\end{exercise}

\begin{exercise}
Verifique a validade da seguinte afirmação:
Dados $f : A \to B$ e $g : B \to C$, se $g \comp f$ é constante então pelo menos uma das $f,g$ também é.
\end{exercise}
