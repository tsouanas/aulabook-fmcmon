\begin{exercise}
\topic{funções!composição}
    Prove que a composição de injeções é uma injeção, a composição de sobrejeções é uma sobrejeção, e consequentemente, a composição de bijeções é uma bijeção.
\end{exercise}

\begin{exercise}
\topic{funções}
    Sejam $A \not = \emptyset$ um conjunto e $f : A \to \mathcal P A$ definida pela equação:
    $$
        f(a) = \set a
    $$
    \begin {enumerate} [(a)]
        \item $f$ é injetora?
        \item $f$ é sobrejetora?
    \end {enumerate}
\end{exercise}

\begin{exercise}
\topic{funções}
    Seja $S$ o conjunto de todos os strings não-vazios de um alfabeto $\Sigma$, com $\card{\Sigma} \geq 2$. Considere a função $f : S \times \set{0,1} \to S$ definida pela:
    $$
        f(w, i) = \left \{ \begin{aligned}
                                &ww &\text{, se } i = 0\\
                                &w' &\text{, se } i = 1
                           \end{aligned}
                  \right.
    $$  

    onde $w'$ é o string reverso de $w$, e onde denotamos a concatenação de strings por justaposição.

    \begin{enumerate}[(a)]
        \item $f$ é injetora?
        \item $f$ é sobrejetora?
    \end{enumerate}
\end{exercise}

\begin{exercise}
\topic{funções}
    Sejam $n \in \nats$, e $I = \set{i \in \nats \st i < n}$. Considere a função $\pi : I \times A^n \to A$ definida por:
    $$
        \pi (i, \alpha) = \text{o iésimo elemento da tupla } \alpha = \alpha_i
    $$

    \begin{enumerate}[(a)]
        \item $\pi$ é injetora?
        \item $\pi$ é sobrejetora?
    \end{enumerate}
\end{exercise}
