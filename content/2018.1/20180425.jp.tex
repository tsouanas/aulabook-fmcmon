\begin{exercise}
Para cada relação, determine quais das propriedades abaixo são satisfeitas
\begin{itemize}
    \item reflexividade
    \item simetria
    \item transitividade
    \item assimetria
    \item antissimetria
    \item irreflexividade
    \item ciclicidade
\end{itemize}

\begin{enumerate}[(a)]
    \item $F : \mathcal{Rel}(P,P)$, onde $P$ é o conjunto das pessoas, e $F(x,y)$ sse $x$ é pai de $y$
    \item $D : \mathcal{Rel}(\nats,\nats)$, onde $D(m,n)$ sse $m$ divide $n$
    \item $I : \mathcal{Rel}(Prop, Prop)$, onde $Prop$ é o conjunto das proposições, e $I(p,q)$ sse $p$ implica $q$
    \item $R : \mathcal{Rel}(\reals^2,\reals^2)$, onde $R(p,q)$ sse $p$ é o reflexo de $q$ pelo eixo $y$
\end{enumerate}
\end{exercise}

\begin{exercise}
È possível uma relação binária ser simétrica e assimétrica ao mesmo tempo? Prove ou refute.
\end{exercise}

\begin{exercise}
Dado $\varepsilon \in (0,1)$, definimos a relação $\eqsym_\varepsilon$ tal que $x \eqsym_\varepsilon y$ sse $|x-y| < \varepsilon$. Essa relação é reflexiva? Simétrica? Transitiva?
\end{exercise}
