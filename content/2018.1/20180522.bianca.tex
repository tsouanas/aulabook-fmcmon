\begin{exercise}
\topic{grupos!subgrupos}
Mostre que $\subgroup$ é uma relação de ordem:
\begin{enumerate}[(1)]
    \item $G \subgroup G$
    \item $K \subgroup H$ \ \& \ $H \subgroup G \implies K \subgroup G$
    \item $H \subgroup G$ \ \& \ $G \subgroup H \implies H = G$ 
\end{enumerate} 
\end{exercise}

\begin{exercise}
\topic{grupos!subgrupos}
Nos próximos exercícios, seja $G$ um grupo abeliano.
\begin{enumerate}[(i)]
    \item Se $H = \setst {x \in G} {x = x^{-1}}$, isto é, $H$ consiste em todos os elementos de $G$ que são seus próprios inversos, prove que $H$ é um subgrupo de $G$.
    \item Seja $n$ um inteiro fixo, e seja $H = \setst {x \in G} {x^n = e}$. Prove que $H$ é um subgrupo de $G$.
    \item Suponha que $H$ e $K$ são subgrupos de $G$, e defina $HK$ da seguinte maneira:
        $$
            HK = \setstt {hk}{$h \in H$ e $k \in K$}
        $$
        Prove que $HK$ é um subgrupo de $G$.
\end{enumerate}
\end{exercise}

\begin{exercise}
\topic{grupos!subgrupos}
\topic{grupos!centro}
O centro de um grupo é o conjunto de todos os elementos de $G$ que comutam como todo elemento de $G$, isto é,
$$
    C = \setstt {c \in G} {$cg = gc$ para todo $g \in G$}
$$
Prove que $C$ é um subgrupo de $G$.
\end{exercise}
