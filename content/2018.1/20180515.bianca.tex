\begin{exercise}
\topic{grupos!ordem}
    Seja $a$ elemento de um grupo $G$. Prove os seguintes:

\begin{enumerate}[(i)]
    \item $o(a) = 1$ sse $a = e$.
    \item Se $o(a) = n$, então $a^{n - r} = (a^r)^{-1}$.
    \item A ordem de $a^{-1}$ é a mesma que a ordem de $a$.
\end{enumerate}

\end{exercise}

\begin{exercise}
\topic{grupos!ordem}
Sejam $G$ grupo, $a \in G$ e $m \in \ints$. Prove que

$$
    a^m = e \iff o(a) \divs m
$$

onde $o(a)$ denota a ordem de $a$ no $G$.
\end{exercise}

\begin{exercise}
\topic{grupos!ordem}
Seja $a$ elemento de ordem finita de um grupo $G$. Prove os seguintes:

\begin{enumerate}[(i)]
    \item Se $a^k = e$, onde $k$ é ímpar, então a ordem de $a$ é ímpar.
    \item Se $a^p = e$, onde $p$ é um número primo, então $a$ tem ordem $p$ ($a \not = e$).
    \item A ordem de $a^k$ é um divisor da ordem de $a$.
    \item Se $o(a) = km$, então $o(a^k) = m$.
\end{enumerate}

\end{exercise}
