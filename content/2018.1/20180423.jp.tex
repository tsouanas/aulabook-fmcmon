\begin{definition}
Dada $f : A \to B$, chamamos de \textbf{núcleo} de $f$ a relação $=_f$ definida por $x =_f y$ sse $f(x) = f(y)$.
\end{definition}

\begin{exercise}
Mostre que, para toda função $f$, seu núcleo é uma relação de equivalência.
\end{exercise}

\begin{exercise}
Caracterize as classes de equivalência dos núcleos das funções abaixo:
    \begin{itemize}
        \item $g : \ints \to \ints$; $g(x) = x^2$
        \item $h : P \to P$ onde $P$ é o conjunto das pessoas; $h(x) = \text{ o pai de } x$
    \end{itemize}
\end{exercise}

\begin{exercise}
Seja $f$ uma função injetiva. O que podemos dizer do núcleo de $f$?
\end{exercise}

\begin{exercise}
Dado $k > 2$ natural, definimos $mod_k : \nats \to \nats$ tal que $mod_k(n) = n \mod k$. Como podemos caracterizar o núcleo de $mod_k$?
\end{exercise}
