\begin{exercise}
\topic{grupos!comutatividade}
\topic{grupos!associatividade}
	Se uma nova adição para números reais, denotada pelo símbolo temporário $\boxplus$, é definida por
	$$
		\alpha \boxplus \beta = 2\alpha + 2\beta,
	$$
	ela é comutativa? É associativa?
\end{exercise}

\begin{exercise}
\topic{grupos!comutatividade}
\topic{grupos!associatividade}
	Se uma nova adição para números reais, denotada pelo símbolo temporário $\boxplus$, é definida por
	$$
		\alpha \boxplus \beta = 2\alpha + \beta,
	$$
	ela é comutativa? É associativa?
\end{exercise}

\begin{exercise}
\topic{grupos!comutatividade}
\topic{grupos!associatividade}
	Se uma operação para inteiros positivos, denotada pelo símbolo temporário $*$, é definida por
	$$
		\alpha * \beta = \alpha^{\beta},
	$$
	ela é comutativa? É associativa?
\end{exercise}

\begin{exercise}
\topic{grupos}
    Cada uma das seguintes é uma operação $*$ em $\reals$. Indique se:
    \begin{itemize}
        \item $*$ é comutativa;
        \item $*$ é associativa;
        \item $\reals$ tem um elemento identidade com respeito a $*$;
        \item todo $x \in \reals$ tem uma inversa com respeito a $*$.
    \end{itemize}

    \begin{enumerate}[(a)]
        \item $x * y = x + y + 1$
        \item $x * y = x + 2y + 4$
        \item $x * y = \abs{x - y}$
    \end{enumerate}
\end{exercise}

\begin{exercise}
\topic{grupos}
    Seja $*$ uma operação definida por
    $$
        (a,b) * (c,d) = (ac, bc + d)
    $$
    no conjunto $\set{(x,y) \in \reals \times \reals \st x \not = 0}$. Proceda como no exercício anterior.
\end{exercise}
