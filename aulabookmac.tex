\usepackage{ifthen}
%\usepackage{newfile}
\usepackage{etoolbox}
\usepackage{environ}
\usepackage{xspace}
\usepackage{stringstrings}
\usepackage{datetime2}
\DTMsetstyle{iso}
\edef\compiletime{\DTMnow}
\newwrite\compiletimefile
\openout\compiletimefile=\jobname.compiletime
\write\compiletimefile{\compiletime}
\closeout\compiletimefile
\usepackage[colorlinks]{hyperref}
\hypersetup{
    citecolor = {citecolor}
}
\usepackage{tikz}
\usepackage[
  backend=bibtex8,
  style=alphabetic,
  url=true,
  doi=true,
]{biblatex}
\usepackage{xcolor}
\definecolor{citecolor}{rgb}{0,0.5,0.25}
\definecolor{homeworkcolor}{rgb}{0.2,0.0,0.6}

% layout
\settocdepth{section}
\setsecnumdepth{chapter}
\def\semester#1{\chapter{#1}}
\def\signature#1{\begin{flushright}\itshape#1\end{flushright}}
% make section numbers stick out to the left
\makeatletter
\def\@seccntformat#1{\protect\makebox[0pt][r]{\csname the#1\endcsname\quad}}
\makeatother

% refs

% index
\def\topic#1{\index{#1}\xspace}

% define lesson on a per-teacher basis
\newcounter{lessoncount}
\newenvironment{lesson}[1]
  {\stepcounter{lessoncount}\begin{section}{#1}}
  {\end{section}}
\def\defteacher#1#2#3{%
  \newtoggle{show#1}
  \toggletrue{show#1}
  \NewEnviron{#1}[1]
  {%
    \iftoggle{show#1} {%
      \begin{lesson}{{##1}~~(\capitalize{#1})}%
        \InputIfFileExists{content/#1mac}{}{}%
        \BODY%
      \end{lesson}
    }{}
  }
}

\def\notesname{Notes}

% notes
\newenvironment{notes}
    {\bigskip\begin{center}\large\bfseries\notesname\end{center}\endgraf}
    {}

\defteacher
    {bianca}
    {Bianca Rodrigues Cesarino}
    {221b.rodrigues@gmail.com}
\defteacher
    {victor}
    {Victor Rafael Santos Silva}
    {vicrafsansil@gmail.com}
\defteacher
    {jp}
    {João Pedro Holanda}
    {jpholanda.prf@gmail.com}


% math macros
% math-related macros

% required packages
\usepackage{amsmath}
\usepackage{amsthm}
\usepackage{amsfonts}
\usepackage{amssymb}
\usepackage{mathrsfs}
\usepackage{stmaryrd}

% define theorem-like environments
\newtheoremstyle{homework}
  {0.5em}% measure of space to leave above the theorem. E.g.: 3pt
  {0.5em}% measure of space to leave below the theorem. E.g.: 3pt
  {\color{homeworkcolor}}% name of font to use in the body of the theorem
  {}% measure of space to indent
  {\color{homeworkcolor}\bfseries}% name of head font
  {.}% punctuation between head and body
  { }% space after theorem head; " " = normal interword space
  {}% Manually specify head
\theoremstyle{definition}
\newtheorem {exercise}              {Exercise}  [chapter]
\newtheorem {definition}[exercise]  {Definição} 
\newtheorem {problem}   [exercise]  {Problem}   
\newtheorem {question}  [exercise]  {Problem}   
\theoremstyle{plain}
\newtheorem {theorem}   [exercise]  {Theorem}   
\newtheorem {lemma}     [exercise]  {Lemma}     
\theoremstyle{homework}
\newtheorem {homework}              {Homework}  [chapter]

% definitions
\def\dterm#1{\textsl{#1}}

% standard sets
\def\nats   {{\mathbb N}}
\def\rats   {{\mathbb Q}}
\def\ints   {{\mathbb Z}}
\def\reals  {{\mathbb R}}
\def\complex{{\mathbb C}}
\def\bools  {{\mathbb B}}

% generic tools
\def\ensuremath#1{\ifmmode#1\else$#1$\fi}
%\def\eqvdots{\mathmakebox[\widthof{${}={}$}][c]{\vdots}}
\def\eqvdots{\ \ \vdots}
\def\eqtype{\mathrel{\,\,\kern-.05em:\,}}

% elementary math
\def\abs#1{\left|#1\right|}
\def\sizeof#1{\left|#1\right|}
\def\frac#1#2{{#1\over#2}}
\def\ntimes{\mathbin{\cdot}}
\def\stimes{\,}
\def\floor#1{\left\lfloor#1\right\rfloor}
\def\ceil#1{\left\lceil#1\right\rceil}
\def\cancel#1{\tikz[baseline] \node [strike out,draw,anchor=text,inner sep=0pt,text=black]{$#1$};}

% combinatorics
\def\arrange#1{Arr(#1)}
\def\perm#1#2{P(#1,#2)}
\def\comb#1#2{C(#1,#2)}
\def\binom#1#2{{{#1} \choose {#2}}}

% set theory
\def\intersection{\cap}
\let\isect=\intersection
\def\Intersection{\bigcap}
\let\Isect=\Intersection
\def\union{\cup}
\def\Union{\bigcup}
\def\symdiff{\triangle}
\def\powerset{{\wp}}
\let\pset=\powerset
\def\univ{U}
\def\surjto{\rightarrow\mathrel{\kern-.8em}\rightarrow}
\def\injto{\rightarrowtail}
\def\bijto{\injto\mathrel{\kern-.8em}\rightarrow}

% more set theory
\def\continuum{\euf{c}}
\def\finord#1{\overline{#1}}
\def\eqc{\mathrel{=_{\namedop{c}}}}
\def\neqc{\mathrel{\not\eqc}}
\def\leqc{\mathrel{\leq_{\namedop{c}}}}
\def\geqc{\mathrel{\geq_{\namedop{c}}}}
\def\gtc{\mathrel{>_{\namedop{c}}}}
\def\ltc{\mathrel{<_{\namedop{c}}}}


% number theory
\def\divides{\mathrel{\mid}}
\def\divideseqwidth{\;\mathrel{\mid}\;}
\let\divs=\divides
\let\knuthcong=\cong
\let\knuthncong=\ncong
\def\cong{\equiv}
\def\ncong{\mathrel{\not\equiv}}
\def\congmod#1#2{#1\quad(\mod {#1})}
\def\modulo#1{\left(\mod{#1}\right)}
\def\ndivides{\mathrel{\nmid}}
\def\gengcd#1{\mathopen{\text{\rm gcd}(}#1\mathclose{)}}
\def\genlcm#1{\mathopen{\text{\rm lcm}(}#1\mathclose{)}}
\def\gcd#1#2{\left(#1,#2\right)}
\def\lcm#1#2{\left[#1,#2\right]}
\def\totsym{\varphi}
\def\tot#1{\mathopen{\totsym}\left({#1}\right)\mathclose{}}

% logic metalang
%\def\iffdf{\mathrel{\overset\triangle\iff}}
\def\iffdf{\mathrel{{\buildrel{\text{\sixss def}} \over {\iff}}}}
\def\iffpseudodf{\mathrel{{\buildrel{\text{\sixss ``def''}} \over {\iff}}}}
\def\iffask{\mathrel{{\buildrel{\text{?}} \over {\iff}}}}
\def\letiff{\mathrel{\;:\Longleftrightarrow \;}}
\def\eqdf{\mathrel{{\buildrel{\text{\sixss def}} \over {=}}}}
\def\eqpseudodf{\mathrel{{\buildrel{\text{\sixss ``def''}} \over {=}}}}
\def\askeq{\mathrel{{\buildrel{\text?} \over {=}}}}
\def\eqass{\mathrel{:=}}
\def\asseq{\mathrel{=:}}
\let\leteq=\eqass
\let\eqlet=\asseq
\def\mland{\mathbin{\;\&\;}}
\def\mlor{\mathbin{\;\text{\rm or}\;}}
\def\nimplies{\smartnot\implies}
\def\lrdir{``$\Rightarrow$''}
\def\rldir{``$\Leftarrow$''}
\def\bidir{``$\Leftrightarrow$''}

% complex numbers
\def\modulus#1{\left|#1\right|}

% group theory
\def\subgroup{\mathrel{\leq}}
\def\normal{\mathrel{\trianglelefteq}}


% write aula count
\def\writeaulacount{%
\newwrite\countfile
\openout\countfile=\jobname.count
\write\countfile{\thelessoncount}
\closeout\countfile
}
